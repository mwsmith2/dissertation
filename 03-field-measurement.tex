\chapter{Magnetic Field Measurement} \label{ch:field}

\section{Magnetic Field Uncertainty Budget}

The \mugmtwo experiment allots \SI{70}{ppb} total uncertainty for $\omega_p$ which predominantly comes from magnetic field measurements.  The uncertainty budget contributions are enumerated in source \cite{e989-tdr} and summarized in table \ref{tab:field-uncertainties}.  Uncertainty constraints highlight relevant details for the next sections which walk through the logistics of magnetic field measurements.

\begin{table}[h]
\label{tab:field-uncertainties}
\caption{Magnetic Field Uncertainty Budget}
\centering
\begin{tabular}{l c}
    \hline
    \multicolumn{1}{c}{Source} & Uncertainty [ppb] \\
    \hline
    Calibration of fixed probes      & 35 \\
    Calibration of trolley probes    & 30 \\
    Trolley measurements of $B_0$    & 30 \\
    Interpolation with fixed probes  & 30 \\
    Muon distribution                & 10 \\
    Inflector fringe field           & -  \\
    Transient external B-fields      & 5  \\
    Other sources                    & 30 \\
    \hline
    Total $\delta \omega_p$          & 70 \\
    \hline
\end{tabular}
\end{table}

\section{Measurement Technique} \label{sec:field-measurement-technique}

The uncertainty allocation for magnetic field determination is entirely systematic.  Therefore, a typical measurement needs to be precise enough that the statistical error on the overall magnetic field determination has negligible influence.  This motivates a stated uncertainty goal of \ppb{10} for each standard field measurement \cite{e989-tdr}.  Commercial magnetometers with the required precision and geometric interface were not available within the E989 financial budget, so researchers at \uw developed a set of custom magnetometers. The magnetometers use pulsed nuclear magnetic resonance (pNMR), a reprise of the measurement technique used in E821.  Using pNMR with a sharp resonance at the expected Larmor frequency of the \gmtwo magnetic field, the magnetometers are able to reach the aforementioned precision goals.

\subsection{Pulsed Nuclear Magnetic Resonance}
A pNMR measurement exploits Larmor precession of protons to produce a signal with magnetic field information embedded in a frequency measurement.  

\subsubsection{Spin Polarization}
Each pNMR device has a volume with a population of quasi-free protons which are manipulated to produce the measurement signal.  The protons in the active volume have a small net spin polarization in the direction of the external field due to the spin-field interaction, $U = -\vec{\mu} \cdot \vec{B}$.  The statistical state of the proton spins due to the spin-field interaction follows from the Boltzmann distribution \cite{landau-stat-mech}. At room temperature (\SI{300}{\kelvin}) in the \gmtwo magnetic field (\bmagic), the average difference between the fraction of spin up and spin down protons using CODATA values for $\mu_p$ and $k_B$ is given by

\begin{equation}
\label{eqn:pnmr-proton-polarization}
\frac{e^{-\mu_z B_z/k_B T} - e^{\mu_z B_z/k_B T}}{e^{-\mu_z B_z/k_B T} + e^{\mu_z B_z / k_B T}} = 4.94 \times 10^{-6},
\end{equation}

\noindent
a surprisingly small, but still very useful net spin polarization.

\subsubsection{Radio Frequency Pulse}
To initiate a pNMR signal, a secondary radio frequency (RF) magnetic field is applied in the orthogonal plane by driving a current through a solenoid surrounding the proton sample.  The RF magnetic field alternates at a frequency approximately equal to the expected spin precession frequency in the given external magnetic field.  Define the magnetic fields to be

\begin{equation}
\label{eqn:pnmr-field-with-rf}
\vec{B}(t) = b_0 \cos{(\Omega t)} \hat{x} + b_0 \sin{(\Omega t)} \hat{y} + B_0 \hat{z}
\end{equation}

\noindent
where $B_0$ is the original magnetic field strength, $b_0$ is the RF field strength, and $\Omega$ is the angular frequency of the RF field.

The process unfolds intuitively in the reference frame rotating at $\Omega = \omega_p$.  In the rotating frame, the effective spin precession frequency from $B'_z$ is $\omega_p - \Omega = 0$, and therefore the effective magnetic field $B'_z = 0$.  On the x-axis, $B_x = b_0$ in the rotating frame, so the RF field rotates the spin polarization vector around the x-axis into the x-y plane.  The RF field is turned off at $T = \pi / (2 \Omega)$, so that the spin polarization vectors only rotates \SI{\pi/2}{\radian} and ends on the y-axis, hence, it is often called a $\pi/2$ pulse.  The essential behavior is similar when the RF frequency and the Larmor frequency do not match exactly.  In the lab frame the process can be understood in terms of a resonance effect due to the RF frequency matching the spin transition frequency.  

\subsubsection{Free Induction Decay (FID)}
After the $\pi/2$ pulse, the polarized protons precess in the x-y plane at a rate proportional to the original vertical field.  The precession signal and the subsequent decay is known as a free induction decay.  The same solenoid used to produce the $\pi/2$ pulse now acts as a pickup coil, and the net proton spin vector in the orthogonal plane produces a changing magnetic flux and thereby an induction current as it precesses.  

The signal attenuates due to two dominant effects.  The first effect is called longitudinal relaxation.  It occurs due to spin alignment with $B_z$, the same spin-field interaction which imparted a net polarization in the first place.  The net spin vector returns to equilibrium at an exponential rate with a time constant usually called $T_1$.  The second effect is called transverse relaxation.  It is essentially decoherence among the collection of proton spins.  Not all spins experience exactly the same magnetic fields, so after the RF pulse they precess at different rates and the net spin vector can decohere.  The rate of decoherence is also exponential, and it is given the time constant $T_2$.  The precession frequency determined by analyzing the pNMR signal can serve as high precision magnetic field proxy. 

\subsubsection{Bloch Equations}
The aggregate spins in the proton sample evolves according to the Bloch Equations (equations \ref{eqn:bloch}), an ordinary differential equation that describes the time evolution of the magnetization vector in external magnetic fields.

\begin{align}
\begin{split}
\label{eqn:bloch}
\frac{dM_x(t)}{dt} & = \gamma (\mathbf{M}(t) \times \mathbf{B}(t))_x - \frac{M_x(t)}{T_2} \\
\frac{dM_y(t)}{dt} & = \gamma (\mathbf{M}(t)\times \mathbf{B}(t))_y - \frac{M_y(t)}{T_2} \\
\frac{dM_z(t)}{dt} & = \gamma (\mathbf{M}(t) \times \mathbf{B}(t))_z - \frac{M_z(t) - M_0}{T_1}
\end{split} 
\end{align}

Consider the case of a perfectly uniform magnetic field, $T_2 \rightarrow \infty$, and a perfect $\pi/2$ pulse. This case is referred to in this document as an ``ideal FID'' waveform, and it can be solved exactly from the Bloch equations.  The resulting equation, \ref{eqn:ideal-fid}, exhibits the primary characteristics of an FID, a sinusoid term from proton precession and an exponential decay envelope from the longitudinal relaxation.  Figure \ref{fig:fid-ideal-waveform} exhibits the waveform of an ideal FID.

\begin{equation}
f(t) = e^{-t/T_1} \sin(\omega t - \phi_0)
\label{eqn:ideal-fid}
\end{equation}

\begin{figure}
\centering
\includegraphics[width=0.5\linewidth]{fig/fid-ideal-waveform}
\caption{
    The figure shows an idealized FID waveform. Functionally, the waveform is a sinusoid multiplied with an exponential decay between at some initial time.  It represents the signal from a perfect $\pi/2$ to a perfect pNMR probe.
    \label{fig:fid-ideal-waveform}
}
\end{figure}

In a realistic situation, the FID signal is a summation of signals produced by protons precessing at different frequencies due to gradients in the magnetic field, and the signal can bear strong deviations from the ideal FID form.  Nevertheless, the ideal signal is a valid testing ground for frequency extraction algorithms.  Examples of measured FIDs are shown in figure \ref{fig:fid-measured-waveforms}.  The measured FIDs exhibit distortions from defects in electronics like the asymmetry of the first waveform and the spikes in the second waveform.  They also exhibit the effects of magnetic field gradients on the signal.  For instance, the second FID plot shows node-like behavior on the waveform which is indicative of an FID of a pNMR probe in magnetic field gradients.

\begin{figure}
\centering
\includegraphics[width=0.9\linewidth]{fig/fid-measured-waveforms}
\caption{
    Two examples of measured FIDs are shown above. The image on left was measured with a test setup at CENPA (University of Washington), and the waveform on the right is an example from the E821 experiment fixed probe system.  The measured FIDs show distortions from defects in electronics like the asymmetry of the first waveform and the spikes in the second waveform.  They also exhibit the effects of magnetic field gradients on the signal.  For instance, the second FID plot shows node-like behavior on the waveform which is indicative of a gradient of FID frequencies beating against each other.
    \label{fig:fid-measured-waveforms}
}
\end{figure}

\subsubsection{Systematic Effects}

There are several systematic effects that probe designs can mitigate. 
The typical set of corrections to pNMR frequency is given in the \mugmtwo TDR \cite{e989-tdr} as

\begin{equation}
\label{eqn:nmr-effects-model}
\omega_{probe} = (1 - \sigma(T) - \delta_b(T) - \delta_p - \delta_s) \omega_p.
\end{equation}

The first correction is due to diamagnetic shielding effects in the protonated volume.  The correction is represented by the $\sigma$ function.  The electron clouds of nearby molecules act as a local dielectric and shield the proton from the full magnetic field.  This effect is typically on the order of tens of ppm.  It is typically modeled with temperature dependence and corrected for in the resulting frequency \cite{schenk-med-nmr}.

The second term, $\delta_b$, denotes the correction for the bulk susceptibility of the sample.  It is also sometimes represented in the form $\left(\epsilon - \frac{4\pi}{3}\right) \chi(T)$.  This includes geometric effects which shift the resonance frequency due to magnetization of the sample.  They are minimized and understood by using ideal shapes with known factors for instance, $\delta_{sphere} = \frac{4\pi}{3}$ and $\delta_{cylinder} = 2\pi$ \cite{e989-tdr}.  Additionally, cylinders and spheres can be fabricated with high accuracy making them common design choices.

The third term, $\delta_p$, corrects for paramagnetic impurities in the sample.  impurities are often added to shorten the signal to the desired time scale.  For instance, $\mathrm{H_2O}$ has a natural signal relaxation time on the order of \SI{10}{\second}, but $\mathrm{H_2O}$ doped with $\mathrm{CuSO_4}$ can reduce the relaxation time to around \SI{40}{\milli\second} \cite{e821-prd}.

The last term, $\delta_s$, includes the effects of magnetic perturbations incorporated into the material design of the system.  The obvious solution is to use materials that are "non-magnetic" even though all materials exhibit some magnetic susceptibility.  Another clever solution is the combine materials in a way that cancels paramagnetic and diamagnetic effects such as copper and aluminum \cite{schenk-med-nmr}.

References \cite{keeler-nmr} and \cite{schenk-med-nmr} are quality resources for more details on NMR fundamentals and systematics respectively.

\subsection{Probe Designs}

The custom pNMR probes come in three different builds.  The standard probes which are used in the fixed probe system and the trolley system for in situ measurements of the magnetic field in the muon storage region.  The absolute calibration probes are used in measurements external to the muons and the storage ring, and the plunging probes are the third type of probes which are used for cross-calibration of the absolute calibration probe and the standard probes.

\subsubsection{Standard pNMR Probes}

The core design of the pNMR probes is the same as the design from the previous experiment. Figure \ref{fig:field-pnmr-probe-design} depicts the design and components of the probe.  Each probe consists of two induction coils, a Teflon backbone, a tunable capacitor, an aluminum shell, and a \SI{5}{\meter} RF cable with SMA connector.  The main induction coil is used to inject the $\pi/2$ pulse, and again to pick up the pNMR precession signal.  A second induction coil is used to match the \SI{50}{\ohm} impedance of the line.  The capacitor allows the probe resonance frequency to be tuned finely.  The aluminum shell provides the outer conductor of a cylindrical capacitor and shields against external effects.

\begin{figure}
\centering
\includegraphics[width=0.9\linewidth]{fig/field-pnmr-probe-design}
\caption{
    The E989 pNMR probe design.  In the center of the diagram, a cylindrical volume holds the petroleum jelly which contains protons to polarize.  The proton volume is surrounded by the serial inductor coil used to inject a the $\pi/2$ pulse and the pick up the FID signal.  The other inductor matches the impedance of the system, and the end of the probe contains a Teflon screw for fine tuning of the resonance.
    \label{fig:field-pnmr-probe-design}
}
\end{figure}

The field team at \uw iterated on the probe design making several improvements.  One of the improvements was in the robustness of the connection to the signal cable.  The new design used a crimp connection to secure the cable to provide a more secure connection.  Another improvement was the design of the tuning capacitor.  With the new design, probe tuning can be done without removing the outer shell.  Removing the shell has a chance of damaging structures inside the probe, and minimizing removals makes tuning easier, faster and safer overall.  Another improvement is a substitution of the material used as a proton sample.  The E821 design used $\mathrm{H_2O}$ as a proton source with $\mathrm{CuSO_4}$ to decrease the $T_1$ relaxation time constant.  The new design replaces the mixture with petroleum jelly as a proton source.  The water design was seen to cause corrosion in many of the E821 probes, so the new design should alleviate those concerns and provide a longer, stable lifetime for the device.

The standard pNMR probes are used in two different but complementary systems to measure the magnetic field, the fixed probe system and the trolley system, which are detailed in section \ref{sec:field-subsystems}.

\subsubsection{Absolute Calibration Probes}

The standard pNMR probes can only give a proxy for the magnetic field, not a well calibrated absolute value.  For the absolute magnetic field value, the absolute calibration pNMR probes have to come into the fold.  The absolute calibration probes work on the same pNMR principles, but the design bears more consideration in minimizing and understanding known systematic effects which shift the precession frequency from that of free protons.  For instance, they may use a precisely fabricated spherical proton volume in order to minimize geometric effects.  And, all probe materials and probe holder materials are stringently tested for magnetic perturbations.

\begin{figure}
\label{fig:field-e821-abs-probe}
\centering
\includegraphics[width=0.9\linewidth]{fig/field-e821-abs-probe}
\includegraphics[width=0.9\linewidth]{fig/field-abs-probe}
\caption{The top image is a design diagram for the E821 absolute calibration probe, and the bottom design diagram is for the new E989 calibration probe.  The E821 probe is still in use alongside the new Macor probe designed by UMass and an entirely different type of probe using $\mathrm{^3He}$ instead of water as a proton source.}
\end{figure}

\subsubsection{Plunging Probes}

The last species of pNMR probe is the so-called plunging probe. The probe is specifically designed to work with a system that transfers the calibration from the absolute probes to the standard probes (shown in figure \ref{fig:field-plunging-probe}).  The plunging probe system deploys a motion controller stage inside a vacuum tight volume (see figure \ref{fig:field-plunging-probe-translation-stage}).  The translation stage aligns the plunging probe with the standard pNMR probes in the trolley system and absolute calibration probes in situ to make frequency comparison measurements.

\begin{figure}
\label{fig:field-plunging-probe}
\centering
\includegraphics[width=0.7\linewidth]{fig/field-plunging-probe}
\caption{A diagram of the plunging probe deployed in E989.  The probe is designed to work with motion system that allows the probe to be align with both the absolute and standard probes to determine the proper shifts needed to bring pNMR measurement from a frequency to an absolute magnetic field value.}
\end{figure}

\begin{figure}
\label{fig:field-plunging-probe-translation-stage}
\centering
\includegraphics[height=14em]{fig/field-plunging-probe-translation-volume}
\includegraphics[height=14em]{fig/field-plunging-probe-translation-stage}
\caption{Images of some plunging probe system hardware.  The apparatus pictured on the left holds the translation stage used to align the plunging probes with other pNMR probes, and the image on the right shows the translation stage itself.}
\end{figure}

\section{Magnetic Field Subsystems} \label{sec:field-subsystems}

\subsection{Fixed Probe System (FPS)}

The fixed probe system is a suite of stationary probes located above and below the muon storage volume.  There are 378 probe in total, and they measure the field continuously while the muon beam is delivered to the storage ring.  In this way, the magnetic field drift is monitored at all times.  The measurement logistics are detailed in figure \ref{fig:fixed-probe-block-diagram}.

\begin{figure}
\label{fig:fixed-probe-block-diagram}
\centering
\includegraphics[width=0.9\linewidth]{fig/fixed-probe-block-diagram}
\caption{The pNMR probes signal requires a parallel system of controlled electronics equipment.  The E989 reprises the NMR pulser design from the E821 experiments. The block diagram illustrates the measurement process and components of the system.  In order to generate a pNMR signal, the first phase is generating a $\pi/2$ pulse to rotate the protonated volume.  The $\pi/2$ pulse is generated by a TTL input trigger which is tunable from \SIrange{4}{7}{\micro\second}.  After the $\pi/2$ pulse, the free induction decay signal is mixed down against a very stable, \SI{61.74}{\MHz} rubidium frequency standard.  The resulting signal feeds through a low-pass filter, and the filtered signal is fed into a waveform digitizer and recorded at a sampling rate of \SI{10}{\MHz} and sampling depth of 16 bits.  The digitized waveform contains frequency information that represents the magnetic field.}
\end{figure}

The fixed probes are arranged in a very specific manner.  They are secured inside grooves cut into the walls of the 12 vacuum chambers that reside inside the storage magnet gap.  In E821, many of the pNMR probe grooves placed the active volume of the probe near pole and yoke boundaries where the magnetic field gradients were large.  Due to the large gradients, the FID signal from the probe was unusable, and the E989 researchers decided to fix the problem.  The grooves were extended \SI{5}{\cm} to shift the pNMR probes away from the magnetic field gradients and improve the FID signal.

Each probe resides at one of three radii: \SI{7082}{\mm}, \SI{7112}{\mm}, or \SI{7142}. The probe arrangement in azimuth repeats over each vacuum chamber effectively following the yoke sections, so yoke B with vacuum chamber 2 is the same azimuthal pattern as yoke C.  The specific angles over the chambers are: \SI{3.72}{\degree}, \SI{6.64}{\degree}, \SI{12.53}{\degree}, \SI{17.53}{\degree}, \SI{22.53}{\degree}, \SI{26.89}{\degree}.  The one exception is yoke A/vacuum chamber 12 where the grooves for the FPS were not extended.  In that case the probes reside at azimuthal angles of: \SI{2.65}{\degree}, \SI{6.64}{\degree}, \SI{12.53}{\degree}, \SI{17.53}{\degree}, \SI{22.53}{\degree}, \SI{27.68}{\degree}.  The last detail in the probe arrangement is the number of probes.  Most vacuum chambers hold 30 probes numbering 3-2-3-2-3-2 at each azimuthal position (top and bottom), but three chambers have 36 probes opting for a 3-3-3-3-3-3 scheme at each position.  Figure \ref{fig:fixed-probe-arrangement} illustrates the arrangement of fixed probes.  Note that the azimuthal spacing and radial spacing is drawn more symmetric in the figure and not to scale.

\begin{figure}
\centering
\includegraphics[height=15em]{fig/fixed-probe-ring-cross-section}
\includegraphics[height=15em]{fig/fixed-probe-arrangement}
\caption{
    The left diagram shows the cross-sectional view of the storage ring with probe locations marked for the trolley and fixed probes.  The three probes above and belows the storage volume illustrate the arrangement of fixed probes.  The right diagram depicts the azimuthal arrangement of fixed probe in the storage ring. The probe arrangement shown is present both above and below the storage volume. Note that the FPS arrangement is not drawn to a realistic geometry.
    \label{fig:fixed-probe-arrangement}
}
\end{figure}

\subsection{Trolley Probe System (TPS)}

The trolley measures the magnetic field in the muon storage volume.  It is a cylindrical shell with wheels that interface with rails protruding from the vacuum chambers which reside in the gap between the pole pieces as seen in figure \ref{fig:fixed-probe-arrangement}.  Internally the trolley contains an array of 17 of the standard pNMR probes as shown in figure \ref{fig:fixed-probe-arrangement} and the full set of electronics needed to produce and digitize FID signals.  It cannot measure while muons are injected though, so there is a naturally trade-off between measuring the magnetic storage field with the trolley and interpolating the field with the fixed probe system while the muons are delivered. In the trolley system the pNMR probes are arranged with one central device, a ring of 4 devices at a radius of \SI{17.5}{\mm}, and 12 more in an outer ring at \SI{3.5}{\mm}.

\begin{figure}
\label{fig:field-trolley-probe-array}
\centering
\includegraphics[width=0.6\linewidth]{fig/field-trolley-probe-array}
\caption{An image of the trolley module itself.  The trolley rides along a rail system inside the vacuum chambers.  It travels through the muon storage volume.}
\end{figure}

\subsection{Fluxgate System (FS)}

The Fluxgate System consists of four vector magnetometers.  The four devices are Bartington Mag690-1000 three-axis magnetic field sensors.  They are capable of measuring at readout rates of 8 kHz but limited in range to \SIrange{-10}{+10}{\gauss}.  The limited range prevents the devices from making measurements too close to the storage field, but that is not their intended use anyway.  The fluxgate sensors serve as a monitor to check for fast transient and periodic magnetic fields in the experimental hall.  The FS picks up magnetic field signals that are too fast for the FPS, and enables the experiment to place a limit on the effect of such transient signals.

\begin{figure}
\label{fig:field-fluxgates}
\includegraphics[height=15em]{fig/field-fluxgate-image}
\includegraphics[height=15em]{fig/field-fluxgate-data}
\caption{The left image is one of the fluxgate sensors.  The right image is an average of measurements of the vertical magnetic field over about an hour.  Similar structure was seen in both the fluxgate system and fixed probe system.}
\end{figure}

\subsection{Absolute Probe System (APS)}

Several unique probes are used as independent determinations of the free proton resonance frequency.  E989 will make absolute field calibration measurements with the E821 water probe, a newly designed water probe, and a $\mathrm{^3He}$ probe. The probes are engineered to best control for systematic effects that shift the free proton frequency.

The original water probe was designed with intent to maximize precision.  It used a highly spherical glass sample volume in order to minimize the bulk susceptibility correction.  Sample impurities were also carefully monitored, and the probe materials were measured for magnetic perturbations.  The measurement achieved an overall precision of \SI{50}{ppb} \cite{e821-abs-probe}.

The redesigned E989 water probe attempts to improve upon some of the systematics of the E821 water probe.  The E989 probe uses more stringent testing of the magnetic susceptibility of probe materials.  The new design also aims to improve machining tolerance for the spherical and cylindrical components of the probe. Stricter monitoring of impurities is also planned by characterizing the $T_1$ and $T_2$ relaxation times of high purity water.  With the improvements, the measurements aims to achieve an uncertainty of \SI{35}{ppb}. \cite{e989-tdr}

The principal motivation for the $\mathrm{^3He}$ probes, a second type of absolute calibration probe, is a cross-check with very different systematics.  For instance, the diamagnetic shielding factor, $\sigma$, on $\mathrm{^3He}$ is larger compared to water probes, but it is known to higher precision.

\begin{alignat}{3}
\sigma_{_{\mathrm{H_2 O}}} & = 25.680(2.5)  & \times 10^{-6} \\
\sigma_{_{\mathrm{^3He}}}  & = 59.967\;43(10) & \times 10^{-6} 
\end{alignat}

\noindent
The diamagnetic shielding also exhibits smaller temperature dependence.  The bulk susceptibility correction, $\delta_b$, is smaller for a $\mathrm{^3He}$ probe due to the proton volume being gaseous.  This also indicates that the measurement is less sensitive to the shape of the proton volume.  The remainder of the design is similar to the water probe, but the materials are different providing another difference in systematics.  Overall the $\mathrm{^3He}$ absolute calibration probe has potential to reach higher precision than the water probe, and the goal for the measurement is again an uncertainty of \SI{35}{ppb}. \cite{e989-tdr}

\subsection{Plunging Probe System (PPS)}

The relative relationship between the absolute calibration probe proton frequency and the probe frequency in the FPS and TPS is done with the plunging probe system.  The plunging probes are brought into the same volume of magnetic field measured and corrected by the absolute probe to establish a calibration from the absolute probe, which does not enter the storage ring, to a system which has a ring entry point.  The plunging probes are inserted into an azimuthal slice of the storage volume.  The plunging probe is moved by a motor stage to be as near to each trolley probe as possible.  In this way, the free proton resonance calibration from the absolute probe can propagate to the standard pNMR probes used to measure the \gmtwo magnetic field in situ.

\begin{figure}
\label{fig:field-plunging-probe-diagram}
\includegraphics[width=0.9\linewidth]{fig/field-plunging-probe-diagram}
\caption{The diagram depicts how the plunging probe correlates pNMR frequencies with the trolley.  The plunging probe is inserted into the calibration volume, and translated to the location of each trolley probe.  The two probes measure the same field and compare frequencies to find the relative calibration correction.}
\end{figure}

\section{Magnetic Field Reconstruction}
The first important aspect of magnetic field reconstruction is understanding that the name itself, magnetic field, is not exactly precise.  The hardware used to make magnetic field measurements mostly consists of pNMR magnetometers.  Measurements using pNMR are proton precession frequencies, not magnetic field measurements.  They act as a proxy for the magnetic field in a specific set of circumstances which are assumed in the \mugmtwo magnetic storage field.  The first assumption is that non-vertical components of the magnetic field are negligible, i.e., $B_r \ll B_z$ and $B_\phi \ll B_z$.  That assumption is validated by radial and azimuthal field measurements, and the nullification of the radial field in the active shimming stage (see chapter \ref{ch:shimming}).  The second assumption is that the pNMR probes are aligned with the magnetic field in such a way that the spin precession frequency that the proton sample experiences is representative of the vertical magnetic field, $B_z$.  This is validated by the same measurements of the radial magnetic field, and by alignment specifications on the probes within their respective device.  The quantity which is truly reconstructed by the magnetic field reconstruction algorithms is the free proton spin precession distribution which acts as a proxy for the magnetic storage field or more precisely the magnetic flux density distribution.

With that caveat, the process of defining the \gmtwo magnetic storage field begins with the trolley system.  The trolley system takes a set of measurements over the whole azimuth of the magnetic storage ring, and that data is used to define the magnetic flux density for any position in the muon storage volume at the specific time of the run, $\omega(\vec{r}, t_i)$. The magnetic field is expressed as a frequency to reflect the pNMR nature of the measurements.  Between trolley runs, measurements from the fixed probes and the fluxgates are used to interpolate change in the flux density distribution.  The changes extracted from the fixed probe data are a function of time only, $\delta \omega_{fps}(t)$.  And, similarly the contributions to the interpolation from the fluxgates can be a function of time only, $\delta \omega_{fs}(t)$.  Putting all the individual contributions together results in equation \ref{eqn:field-construction}.

The algorithms for magnetic field construction are not manifest, and they must be designed and refined by the field team over the duration of the experiment.  In light of the fact that the experiment is currently in the commissioning stage, the contributions cannot be discussed in more than general terms.  The magnetic flux density distribution is re-measured periodically, and only the most recent one contributes to the field definition. The most general form for the magnetic flux density distribution is then

\begin{equation}
\label{eqn:field-construction}
\omega(\vec{r}, t) = \sum_n \Theta(t - t_n) \\
\left[\omega_{n}(\vec{r}) + \delta \omega(t) \right]
\end{equation}

\noindent
where $\omega_{n}(\vec{r})$ is the base magnetic field term over the 3D storage volume at time $t_n$, $\delta \omega(\vec{r}, t)$ is an interpolation term to improve precision between runs, and $\Theta$ is the Heaviside function.  The last step is calibrating the magnetic field determined by the trolley using the plunging probes and the absolute probes. Calibration shifts the flux density measured into terms of the free proton spin precession frequency.

\begin{equation}
\label{eqn:field-calibration}
\omega_0(\vec{r}, t) = (1 - \delta) \omega(\vec{r}, t)
\end{equation}

\noindent The measurement data and algorithms together define the calibrated magnetic flux density distribution over the muon storage volume.

\subsection{Map of the Muon Storage Volume}

The latest data from a trolley ``full run'', a set of measurements encompassing all azimuthal locations in the storage ring taken over a short period of time, serves as a fundamental definition of the field.  The trolley continuously cycles through the 17 probe array while taking measurements at \SI{34}{\Hz}, two full cycles per second.  The full run process takes over an hour which yields at least 7,200 measurements with the full array. Each field measurement is accompanied by multiple location measurements accurate to about \SI{1}{\mm}.  Data from a full set of trolley measurements, $\vec{\omega}_{tps}$ and $\vec{\theta}$, then feeds into an algorithm, $f$, which returns the magnetic field at a location $\vec{r}$ as in equation \ref{eqn:field-3d-trolley}.  The algorithm represents the 3-dimensional magnetic storage field.  The allotted uncertainty for constructing $\omega_{n}(\vec{r})$ is \SI{30}{ppb}.

\begin{equation}
\label{eqn:field-3d-trolley}
\omega_{n}(\vec{r}) = f(\vec{r}, \vec{\omega}_{tps}, \vec{\theta})
\end{equation}

\subsection{Interpolation}

Between trolley runs, the magnetic field still needs to be well known.  The uncertainty budget on the interpolation of the 3D field is \SI{30}{ppb}.  There are two subsystems which contribute to the field interpolation.

\begin{equation}
\label{eqn:field-interpolation}
\delta \omega(t) = \delta \omega_{fps}(\vec{f}, t_i) \\
+ \delta \omega_{fs}(\vec{x}, \vec{b}, t_i)
\end{equation}

The FPS provides one data source used for interpolation.  There are 378 fixed probes around the ring which measure the field at \SI{\sim 1}{\Hz}.  A pNMR frequency is extracted from each probe, and the data is timestamped.  Some algorithm, $f$, for field interpolation will be designed using all fixed probe measurements taken since the last trolley run, such as

\begin{equation}
\label{eqn:field-interpolation-fps}
\delta \omega_{fps}(t) = \sum_{t_i = t_n}^{t_i < t} f(\vec{\omega}_i, t_i)
\end{equation}

The second arm of magnetic field interpolation is the fluxgate magnetometers.  The FS measures the transient magnetic fields at high rates, \SI{\sim 1}{\kHz}.  There are three fluxgate sensors in the array which can be moved around the ring to monitor local magnetic field transients at multiple locations.  The transient magnetic fields measured in the fluxgates, $\vec{b}$, need to be well correlated and calibrated with the magnetic fields measured in the fixed probes, but they are a potentially powerful tool for discerning transients in the field at moderate, common frequencies, such as \SI{15}{\Hz} or \SI{60}{\Hz}.  Each sensor needs a human defined probe position vector denoted by $\vec{x}_i$, and provides a vector magnetic field value, $\vec{b}_i$, and timestamp, $t_i$.  If the transient fields are small, as they are expected to be, the fluxgate will be used to place a limit on the effect of transient fields.

\begin{equation}
\label{eqn:field-interpolation-fm}
\delta \omega_{fs}(t) \\
= \sum_{t_j = t_n}^{t_j < t} f(\vec{x}_i, \vec{b}_i, t_j)
\end{equation}

\subsection{Calibration}

Calibration is the final step in producing an absolute magnetic field map.  The calibration flows down from the absolute probe in two stages.  The absolute calibration probe and the plunging probe both measure the field in a certain location of a monitored and understood field.  The relationship is parameterized by a chemical shift model as given in equation \ref{eqn:field-calibration-abs-pp}.

\begin{equation}
\label{eqn:field-calibration-abs-pp}
\omega_{aps} = (1 - \delta_{pps \rightarrow aps})\omega_{pps}
\end{equation}

The second stage of calibration is similar to the first.  The plunging probe measures the field in approximately the same volume as the trolley probes.  The calibration relationship can then be transferred in a similar way.  Equation \ref{eqn:field-calibration-pp-fp} is given as an example.

\begin{equation}
\label{eqn:field-calibration-pp-fp}
\omega_{pps} = (1 - \delta_{tps \rightarrow pps})\omega_{tps}
\end{equation}

Bringing the magnetic field map and the calibration function together, one arrives at an expression for the magnetic field map over time.

\begin{equation}
\label{eqn:field-omega-p}
\omega_0(\vec{r}, t) = (1 - \delta_{pps \rightarrow aps}) \\
(1 - \delta_{tps \rightarrow pps}) \sum_n \Theta(t - t_n) \\
\left[\omega_{ts,n}(\vec{r}) + \delta \omega_{fps}(t) + \delta \omega_{fs}(t)\right]
\end{equation}

\noindent
This is the culmination of all magnetic field measurements and analysis.  It is referred to as the magnetic field map, but more precisely it is the free proton spin precession distribution as a function of time inside the storage volume. 

\subsection{Beam Convolution}

The final step in determining the quantity which goes into the expression for $a_\mu$, equation \ref{eqn:muon-g-2}, is convoluting the proton precession distribution with the muon beam distribution.  The quantity really needs to represent the average magnetic field experienced by the stored muons.  The muon distribution is measured by Fiber Harp Detectors and reconstructed using the Tracker Detectors (see \cite{e989-tdr} for more detail).  The muon-weighted free proton precession frequency can be determined by convolving the calibrated magnetic flux density distribution, $\omega_0$, with the muon density distribution, $M$.

\begin{equation}
\label{eqn:field-omega-p-avg}
\langle \omega_p \rangle = \iint M(\vec{r}, t) \, \omega_0(\vec{r}, t)\; dt\,dV
\end{equation}

\chapter{Field Measurement}

\section{Measurement Technique}

The \gmtwo experiment has stringent requirements on the measurement uncertainty of the magnetic field value.  Stringent enough to aim for an entire error budget on the field uncertainty which is systematic.  Each individual measurement needs to be precise enough that the statistical error of the combined field value is negligible.  This puts an uncertainty limit of around \ppb{10}.  The field measurement technique was custom pulsed proton nuclear magnetic resonance (pNMR), a reprise of the measurement technique used in E821.  The probes were designed to have a very high Q-value resonance at the expected resonance frequency of \SI{61.79}{\MHz}.

\subsection{Proton NMR Signal}

The essence of a pNMR measurement is frequency measurement of another type of precession, proton precession.  The source of the pNMR signal is a volume of material with a population of quasi-free protons.  The active volume is polarized in the direction of a large field, in the case of \gmtwo a \bmagic field in the vertical direction.  A secondary field rotating in the orthogonal plane at a frequency approximately equal to the expected precession frequency of B-magic.  The intended effect of the secondary field is intuitively understood in the rotating reference frame.  In the rotating frame the vertically (z-direction) polarized protons experience an orthogonal (y-direction) field which rotates them into the orthogonal plane (x-y plane).  In the lab frame the polarized protons continue to precess at a rate proportional to the strong vertical field.  Precession in the orthogonal plane produces an induction signal in a nearby pickup coil which is subsequently digitized and analyzed.  A result of robust frequency analysis on the pNMR signal serves as high precision field proxy for \gmtwo experiment.

\subsection{Probe Design}
\todo{finish on hw design}
The core design of the pNMR probes was based on the design from the previous experiment. Each probe consists of two pickup coils, a teflon backbone, a tunable capacitor, an alimunimum shell, and a \SI{5}{\meter} BNC cable.  One coil is used to inject the $\pi/2$ pulse which rotates the protons into the orthogonal plane, and the other is the induction coil used to produce the pNMR precession signal.  The capacitor allows the probe resonance to be tuned finely.  The aluminum shell allots some capacitance to the probe and shields against external effects.

\todo{add schematic of pNMR probes}

The field team at \uw iterated on the probe design making several improvements.  One of the improvements was in the robustness of the connection to the BNC cable.  The new design used a crimp connection to secure the cable to provide a more robust connection.  Another improvement was the design of the tuning capacitor.  With the new design, the tuning of the probe can be done without removing the outer shell.  Removing the shell the has the chance of damaging the innards of the probe, and minimizing removals makes tuning easier, faster and safer overall.  Another major improvement is the material used as protonated, polarized volume.  The new design replaced the water and copper sulfate volume with a petroleum jelly.  The water design was seen to cause corrosion in many of the E821 probes, so the new design should alleviate those concerns and allot the provide a longer, stable lifetime.

\todo{Get resources from Martin, have him proofread}

\todo{Maybe add a paragraph on the nuances of petroleum jelly}

The pNMR probes are used in two fairly different systems to measure the field.  The \fps comprises 378 probes located all around the storage ring to achieve thorough field coverage.  The actual volumes occupied by the fixed probes are outside of the muon storage volume.  The purpose of the \fps is to monitor the drift of the field at different locales around the ring.  The other system to use the pNMR probes is the Trolley System. The Trolley contains an array of 17 probes which carve out an azimuthal plane in the muon storage volume.  The trolley runs on rails all around the ring and uses the pNMR probes to measure the magnetic field in the muon storage region.  

\subsection{Fixed Probe System Design}

The pNMR probes signal requires a parallel system of controlled electronics equipment.  The E989 reprised the NMR pulser design from the E821 experiments.  In order to generate a pNMR signal, the first phase is generating a $\pi/2$ pulse to rotate the protonated volume.  The $\pi/2$ pulse is generated by TTL input trigger which is tunable from \SIrange{4}{7}{\micro\second}.  After the the $\pi/2$ pulse, the free induction decay signal is mixed down against a very stable, \SI{61.74}{\MHz} rubidium frequency source.  The resulting signal feeds through a lowpass filter with $f_0$ of around \SI{200}{\kHz} \todo{check value}.  The filtered signal is fed into a waveform digitizer and recorded at a sampling rate of \SI{10}{\MHz} and samplingt depth of $16\;bits$.  The digitized waveform contains frequency information that represents the magnetic field.

\todo{signal diagram of the fixed probe system}

\subsection{Trolley System Design}

\todo{get resources on this if needed}

\subsection{Absolute Calibration}

The frequencies measured by the pNMR probe systems in the ring are used to construct a 3D map of proton frequencies in the muon storage volume.  The frequency is a direct proxy for the field, but not an absolute value.  To bring the measurement back to an absolute proton precession precession, the absolute field value, there must be a correction applied.  The size of that correction is determined by the absoluate calibration system.

Several unique probes are used as independent determinations of the free proton resonance frequency.  E989 will make absolute field calibration measurements with the E821 probe, a newly design water probe from UMass, and a $\mathrm{^3He}$ probe with very different systematics. The probes are designed to minimized systematic effects that shift the free proton frequency.  

\todo{add free proton + systematic effects eqn}

The model for free proton frequency includes corrections for geometric effects of the probe structure.  These geometric effects couple into the overall capacitance of the probe which can shift the resonance frequency.  They are minimized by using shapes with known ideal factors, cylinders and spheres.  The overall probe design must also minimize the magnetic perturbations from ferromagnetic and paramagnetic materials.  As such each component is vetted independently before becoming a part of the absolute probe.  The measurement also needs to control the environment of the probe and make a temperature correction.

\todo{ask d-flay and midhat if this is general or should split into he3/water probes}

The relative relationship between the absolute probe proton frequency and the pNMR probe frequency is done the so-called plunging probe system.  The plunging probes are brought into the same volume of field measured and corrected by the absolute probe to establish a transfer calibration from the absolute probe, which does not enter the storage ring, to a system which has a ring entry point.  The plunging probes are inserted into an azimuthal slice of the storage volume, plunged if you will.  The plunging probe is moved by a motor stage to be as near to each trolley probe as possible.  In this way, the free proton resonance from the absolute probe can propagate to the pNMR monitor probes used to measure the \gmtwo magnetic field.

\section{NMR Analysis}

The signal generated in the pNMR probes contains frequency information that represents the effective field seen by the probe's protonated \note{might be wrong word} volume.  A fraction of the polarized protons rotate into orthogonal plane, then precess at a frequency proportional to the dominant field that had previously polarized the particles.  The average frequency of the the precessing protons immediately after the $\pi/2$ pulse is target of analysis.  That value represents the magnetic field in the volume.  The digitize precession signal recorded as the protons slowly rotate back into the original polarization and decohere is called the Free Induction Decay (FID).

\todo{insert equation for macro frequency of protons}

\subsection{Free Induction Decays}

The polarized protons involved in the FID evolve according to the Bloch Equations.  An ordinary differential equation that couples magnetization and field in different dimensions.

\begin{equation}
\frac{dM_x(t)}{dt} = 
\gamma (\mathbf{M}(t) \times \mathbf{B}(t))_x - \frac{M_x(t)}{T_2}
\label{eqn:bloch-x}
\end{equation}

\begin{equation}
\frac{dM_y(t)}{dt} = 
\gamma (\mathbf{M}(t)\times \mathbf{B}(t))_y - \frac{M_y(t)}{T_2}
\label{eqn:bloch-y}
\end{equation}

\begin{equation}
\frac{dM_z(t)}{dt} = 
\gamma (\mathbf{M}(t) \times \mathbf{B}(t))_z - \frac{M_z(t) - M_0}{T_2}
\label{eqn:bloch-z}
\end{equation}

The ideal FID waveform from a pure field value and perfect $\pi/2$ pulse can be solved exactly from the differential equations.  The resulting equation, \ref{eqn:ideal-fid}, exhibits the primary characteristics of an FID, a sinusoid term representing the proton precession and an exponential decay envelope representing the $T_2$ relaxation.  Figure \ref{fig:ideal-fid} exhibits the waveform of an ideal FID.

\begin{equation}
FID(t) = e^{-t/T_2} \sin(\omega t - \phi_0)
\label{eqn:ideal-fid}
\end{equation}

\begin{figure}
\todo{insert ideal FID figure}
\label{fig:ideal-fid}
\end{figure}

In a realistic situation, the FID signal is a summation over varying field values, and the signal bears strong deviations from the ideal FID form.  Nevertheless, the ideal signal is a valid testing bed for frequency extraction algorithms.

\subsection{Frequency Extraction Methods}

As in most data analysis, the first step is cleaning and characterizing the data.  In the case of FIDs, the process involves removing the baseline \note{and possibly trend} and determining the start time and stop time of useful FID signal within the frame of the entire recorded waveform.  The baseline value is calculated using initial segment of the waveform.  The signal range is computed using the maximum amplitude around the signal baseline and applying a threshold on to find the first and last sections of the waveform with an envelope above that threshold.

\subsubsection{Zero Crossing}
The simplest method for determining the FID frequency involves counting zero crossings of the signal.  
\todo{error analysis}

\subsubsection{Spectral Centroid}
Another straightforward technique employed to extract the FID frequency relies on the spectral density of the signal. 

\begin{equation}
FFT(\omega_k) = \sum_{n=0}^{n=N} e^\frac{-i \omega_0}{2\pi k} f(x_n)
\label{eqn:fid-fft}
\end{equation}

\begin{equation}
PSD(\omega_k) = |FFT(\omega_k)|^2
\label{eqn:fid-psd}
\end{equation}

In the power spectral density (PSD), the FID frequency manifests as a peak energy.  The frequency is computed as a frequency weighted  sum of bins symmetrically around the peak value.

\begin{equation}
\omega_{CN} = \sum_{i=i_{max} - N/2}^{i=i_{max} + N/2} PSD(\omega_i)
\label{eqn:freq-cn}
\end{equation}

\todo{error analysis}

\subsubsection{Lorentzian Peak}
Peak fitting routines can improve precision upon the centroid technique.  Peak fitting works well using a Lorentzian Distribution, Eqn. \ref{eqn:lorentzian} around the maximum frequency bin.

\begin{equation}
F(\omega) = \frac{1}{\pi}\frac{\frac{1}{2} \Gamma}{(\omega - \omega_0)^2 + (\frac{1}{2} \Gamma)^2}
\label{eqn:lorentzian}
\end{equation}

\subsubsection{Analytical Form Fit}
The full analytical form of the idea FID PSD peak is trickier to implement

\todo{maybe cut this section}

\subsubsection{Polynomial Phase Fit}
The most intricate frequency determination technique is involves calculating the complementary phase of the original signal.  Given the knowledge that the original signal is harmonic \note{need to validate this}, the Hilbert Transform (Eqn. \ref{eqn:hilbert}) gives the imaginary phase of a signal.

\begin{equation}
H(t) = IFFT(-i \mathrm{sgn}(\omega) \cdot FFT(F(t)))
\label{eqn:hilbert}
\end{equation}

\begin{equation}
\phi(t) = \arctan(H(t) / F(t))
\label{eqn:phase}
\end{equation}

With the phase as a function of time, the FID frequency manifests as the linear slope of the data.  In fact, the phase information yields frequency as a function of time.  The frequency of interest is the frequency immediately after the $\pi/2$ pulse.

\todo{error analysis}

\subsection{Frequency Extraction Efficacy}

Each FID frequency extraction technique needs to be tested and validated.  A whole slew of data options can be used: ideal functional FIDs, simulated FID, E821 waveforms, E821 probes in a local \uw magnetic.  Each dataset presents a test bed for determining the precision, accuracy and failure modes of each frequency extraction technique.

\subsubsection{Ideal FIDs}
The simplest case, the frequency is static in time and all techniques should perform well.  If the technique cannot handle the simplest FIDs, then it will not be worth using to analyze the E989 data.  The effectiveness was testing on 10000 FIDs using a seed frequency of 47 kHz, near the expected nominal pNMR frequency of 50 kHz.  

\begin{figure}
    \label{fig:ideal-fid-spectral}
    \includegraphics[width=0.45\linewidth]{fig/ideal-fid-cn}
    \includegraphics[width=0.45\linewidth]{fig/ideal-fid-lz}
    \caption{Centroid Method and Lorentzian Method}
\end{figure}

\begin{figure}
    \label{fig:ideal-fid-time-domain}
    \includegraphics[width=0.45\linewidth]{fig/ideal-fid-ph1}
    \includegraphics[width=0.45\linewidth]{fig/ideal-fid-sn}
    \caption{Linear Phase Method and Sinusoid Method}
\end{figure}

\note{verify results, combine results into single image}

\subsubsection{Simulated FIDs}
Integrating the Bloch Equations is a clear path to simulate FID waveforms.  The waveforms are similar to those to the ideal FIDs with some possible effects from a deviation from non-ideal, finite $\pi/2$ pulses frequency from the Larmor frequency.  The simulation uses a lower Larmor frequency of around \SI{1}{\MHz} as opposed to the real Larmor frequency of \SI{61.79}{\MHz}.  The reason being that it is more efficient to simulate at lower frequencies, because the integration step can be much larger.  Additionally, the entire system runs through a lowpass filter at \SI{200}{\kHz}, so all the higher frequency content is culled from the final FID signal.

\begin{figure}
    \label{fig:sim-fid-spectral}
    \includegraphics[width=0.45\linewidth]{fig/sim-fid-cn}
    \includegraphics[width=0.45\linewidth]{fig/sim-fid-lz}
    \caption{Centroid Method and Lorentzian Method}
\end{figure}

\begin{figure}
    \label{fig:sim-fid-time-domain}
    \includegraphics[width=0.45\linewidth]{fig/sim-fid-zc}
    \includegraphics[width=0.45\linewidth]{fig/sim-fid-ph1}
    \caption{Linear Phase Method and Sinusoid Method}
\end{figure}

\todo{summary table}

\subsubsection{Simulated Gradient FIDs}
An expected problem complication with actual magnetic field measurements arises from the presence of gradients in the field.  Any real magnetic field will have gradients over the volume of the probe which can distort the signal of the probe.  These effects can be mimicked using a superposition of simulated FIDs over a small range.  One notices node-like behavior in the envelope when gradients are present (and E821 waveforms exhibited this behavior).  The simulations aim to tests the effects of small gradients on the FID frequency extraction precision and accuracy.

The range of magnetic field gradient over the pNMR probes anticipated in the \gmtwo storage field are on the order of SIrange{10}{100}{ppb} \note{check if ppb or ppm}. A collection of simulated FIDs was created using the Bloch Equation integrator with a range chosen to be \SI{47}{\kHz} \SI{\pm 1000}{ppb}. To implement a gradient, FIDs from the simulation collection were put into a weighted sum in a way that did not change the average frequency just the variation of over the probe.

\begin{figure}
\label{fig:sim-gradient-all-fids}
\includegraphics[width=0.9\linewidth]{fig/sim-gradient-all-fids}
\caption{A collection of FID waveforms simulated by integrating the Bloch Equations.  The step size between frequencies was chosen to be \SI{0.1}{ppb}, so that the variation between FIDs was smooth and a sum of waveforms could be used to approximate an integral.}
\end{figure}

\begin{figure}
\label{fig:sim-gradient-lin-grad}
\includegraphics[width=0.9\linewidth]{fig/sim-gradient-lin-grad-0ppm}
\includegraphics[width=0.9\linewidth]{fig/sim-gradient-lin-grad-100ppm}
\includegraphics[width=0.9\linewidth]{fig/sim-gradient-lin-grad-200ppm}
\caption{Example simulation waveforms made using a linear gradient sum over frequencies.  The beat frequencies emplace nearly total nodes onto the waveform envelope.}
\end{figure}

\begin{figure}
\label{fig:sim-gradient-quad-grad}
\includegraphics[width=0.9\linewidth]{fig/sim-gradient-quad-grad-0ppm}
\includegraphics[width=0.9\linewidth]{fig/sim-gradient-quad-grad-100ppm}
\includegraphics[width=0.9\linewidth]{fig/sim-gradient-quad-grad-200ppm}
\caption{Example simulation waveforms made using a quadratic gradient sum over frequencies.  The beat frequencies impart softer waists onto the waveform envelope compared to the purely linear gradient.}
\end{figure}

\todo{redo gradient analysis effects.}

\subsubsection{FID Measurements}
A small set of fully digitized waveforms from E821 was also available to cut the FID algorithms' teeth on.  Testing against this dataset had a real difference in that the true precession frequency was not known.  One can still build distributions to get a sense for the precision and stability of a analysis method.

\todo{dig up or redo}

\subsection{FID Frequency Results}

\todo{really depends on results that I insert above}

\documentclass[11pt, proquest] {uwthesis} [2017/04/14]

\usepackage{color}
\usepackage{xspace}
\usepackage{graphicx}
\usepackage{siunitx}
\usepackage{amsmath}

\DeclareMathOperator*{\argmin}{argmin}
\DeclareSIUnit\gauss{G}

\begin{document}

% Functions
\newcommand{\todo}[1]{\textbf{\color{red}{TODO: #1}}}
\newcommand{\note}[1]{\textbf{\color{yellow}{(#1)}}}

% Places
\newcommand{\tsm}{the Standard Model\xspace}
\newcommand{\smpp}{the Standard Model of Particle Physics\xspace}
\newcommand{\uw}{The University of Washington\xspace}

% Lengths
\newdimen\fdh
\fdh=8em % Height of a single Feynman Diagram

% g-2 helpers
\newcommand{\gmtwo}{$g\hbox{--}2$\xspace}
\newcommand{\mugmtwo}{$(g\hbox{--}2)_\mu$\xspace}
\newcommand{\wa}{$\omega_a$\xspace}
\newcommand{\fps}{Fixed Probe System\xspace}

\newcommand{\bmagic}{\SI{1.4513}{\tesla}\xspace}
\newcommand{\rmagic}{\SI{7.112}{\meter}\xspace}
\newcommand{\pmagic}{\SI{3.094}{\GeV/c}\xspace}
\newcommand{\gev}[1]{\SI{#1}{\GeV/c}\xspace}
\newcommand{\ppm}[1]{\SI{#1}{ppm}\xspace}
\newcommand{\ppb}[1]{\SI{#1}{ppb}\xspace}

\setcounter{tocdepth}{1}  % Print the chapter and sections to the toc

\prelimpages

\Title{Developing the Precision Magnetic Field for the E989 Muon \gmtwo Experiment}
\Author{Matthias W. Smith}
\Year{2017}
\Program{Physics}

\Chair{David Hertzog}{Professor of Physics, g-2 Co-Spokesperson}{Physics}
\Signature{Boris Blinov}
\Signature{Yen-Chi Chen}
\Signature{Jason Detwiler}
\Signature{Blayne Heckel}
\Signature{Steven Sharpe}

% \copyrightpage

\titlepage  

\setcounter{page}{-1}
\abstract{%
The experimental value of \mugmtwo historically has been and contemporarily remains an important probe into \tsm and proposed extensions.  Previous measurements of \mugmtwo exhibit a persistent statistical tension with calculations done using \tsm which implicates that the theory is incomplete and constrains possible extensions.  E989, The Fermilab Muon g-2 experiment endeavors to increase the precision over previous experiments and continue to probe deeper into the tension with \tsm.  The \mugmtwo experimental implementation facilitates frequency measurements of two spin precession values defined by the magnetic field, proton precession and muon precession.  The precision on magnetic field measurements and the overall magnetic field uniformity achieved over the muon storage volume are then two undeniably important aspects of the experiment.  The current thesis details the methods employed to achieve magnetic field goals and results of the effort.
}

\tableofcontents
%\listoffigures
%\listoftables  % I have no tables

\acknowledgments{% \vskip2pc
  % {\narrower\noindent
I thank my parents for stoking my scientific curiosity and consistent love and support.  Many thanks to my teachers, professors, mentors, and advisors for funneling that scientific curiosity toward the tangible, lofty goals which help push the boundaries of human knowledge.  And, I would like to thank all of my friends for the special blend of support and reprieve that they afforded.
  % \par}
}

% end of the preliminary pages
 
\textpages
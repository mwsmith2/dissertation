\setcounter{tocdepth}{1}  % Print the chapter and sections to the toc

\prelimpages

\Title{Developing the Precision Magnetic Field for the E989 Muon \gmtwo Experiment}
\Author{Matthias W. Smith}
\Year{2017}
\Program{Physics}

\Chair{David Hertzog}{}{Physics}
\Signature{Blayne Heckel}
\Signature{Steven Sharpe}

\titlepage

\newpage
\vspace{7em}
\begin{center}
\textcopyright Copyright 2017

Matthias W. Smith
\end{center}
\newpage
\relax\relax

\setcounter{page}{-1}
\abstract{%
The experimental value of \mugmtwo historically has been and contemporarily remains an important probe into \tsm and proposed extensions.  Previous measurements of \mugmtwo exhibit a persistent statistical tension with calculations using \tsm implying that the theory may be incomplete and constraining possible extensions.  The Fermilab Muon g-2 experiment, E989, endeavors to increase the precision over previous experiments by a factor of four and probe more deeply into the tension with \tsm.  The \mugmtwo experimental implementation measures two spin precession frequencies defined by the magnetic field, proton precession and muon precession.  The value of \mugmtwo is derived from a relationship between the two frequencies.  The precision of magnetic field measurements and the overall magnetic field uniformity achieved over the muon storage volume are then two undeniably important aspects of the experiment in minimizing uncertainty.  The current thesis details the methods employed to achieve magnetic field goals and results of the effort.
}

\tableofcontents
%\listoffigures
%\listoftables  % I have no tables

\acknowledgments{% \vskip2pc
  % {\narrower\noindent
I thank my parents for stoking my scientific curiosity and consistent love and support.  Many thanks to my teachers, professors, mentors, and advisors for funneling that scientific curiosity toward the tangible, lofty goals which help push the boundaries of human knowledge.  And, I would like to thank all of my friends for the special blend of support and reprieve that they afforded.
  % \par}
}

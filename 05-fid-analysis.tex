\chapter{FID Analysis}

The signal generated in the pNMR probes contains frequency information that represents the effective field seen by the probe's protonated volume.  A fraction of the polarized protons rotate into orthogonal plane, then precess at the Larmor frequency.  The average frequency of the the precessing protons immediately after the $\pi/2$ pulse is target of analysis.  That value represents the magnetic field in the volume.  The digitized precession signal recorded as the protons slowly rotate back into the original polarization and decohere is called the Free Induction Decay (FID).

\section{Frequency Extraction Methods}

As in most data analysis, the first step is cleaning and characterizing the data.  In the case of FIDs, the process involves removing the baseline (and trend in some cases) and determining the start time and stop time of useful FID signal within the frame of the entire recorded waveform.  The baseline value is calculated using initial segment of the waveform.  The signal range is computed using the maximum amplitude around the signal baseline and applying a threshold on to find the first and last sections of the waveform with an envelope above that threshold.

\subsection{Zero Crossing}
The simplest method for determining the FID frequency involves counting zero crossings of the signal.  The technique was used in the previous \mugmtwo experiment in hardware, and has been re-implemented in software for E989.  The crux of the technique is to count all zero crossings in the signal, then, determine the time of the first and last zero crossing using a polynomial fit.  The method prevents double counting of zeros by requiring the signal go above a threshold before allowing another countable zero crossing.  The hysteresis constraint is illustrated in figure \ref{fig:fid-freq-zc-hysteresis}.  The frequency of the FID is then given by

\begin{equation}
\label{eqn:fig-freq-zc}
\omega_{zc} = \frac{N_{zc}}{2\pi(t_f - t_i)}
\end{equation}

\noindent 
where $N_{zc}$ is the number of zero crossings counted by the algorithm.

\begin{figure}
\label{fig:fid-freq-zc-hysteresis}
\includegraphics[width=0.9\linewidth]{fig/fid-freq-zc-hysteresis}
\end{figure}

An uncertainty analysis can be examined using a variance analysis given in equation \ref{eqn:uncertainty-analysis} \todo{need reference}.

\begin{equation}
\label{eqn:uncertainty-analysis}
\frac{\delta f}{f} \approx \frac{1}{f} \sum_{i = 1} \frac{\partial f}{\partial x_i} \delta x_i
\end{equation}

\noindent
In the case of the zero crossing algorithm, the uncertainty analysis yields

\begin{equation}
\label{eqn:fid-ferr-zc}
\frac{\delta \omega}{\omega} = \frac{\delta N}{N} \\
+ \frac{\delta \Delta T}{\Delta T}.
\end{equation}

\noindent
For the first term in \ref{eqn:fid-ferr-zc}, larger values of N lower the uncertainty, and for the second term longer FID waveforms suppress the uncertainty.  By using a proper hysteresis threshold, $\delta N$ can be suppressed to be essentially zero. It is informative to examine the uncertainty within typical limit values for the pNMR probe frequency.  For limits on acceptable signals, an FID can be defined to have a length of \SI{100}{\micro \second} and a frequency of \SI{20}{\kHz}.  In this case, the signal is expected to exhibit $\delta \omega / \omega \approx 5 \times 10^{-6}$. 

\todo{maybe write up more thorough error analysis appendix}

\subsection{Spectral Centroid}
Another straightforward technique employed to extract the FID frequency relies on the spectral density of the signal. 

\begin{equation}
\label{eqn:fid-fft}
FFT(\omega_k) = \sum_{n=0}^{n=N} e^\frac{-i \omega_0}{2\pi k} f(x_n)
\end{equation}

\begin{equation}
\label{eqn:fid-psd}
PSD(\omega_k) = |FFT(\omega_k)|^2
\end{equation}

In the power spectral density (PSD), the FID frequency manifests as a peak energy.  The frequency is computed as a frequency weighted sum of bins symmetrically around the peak value.

\begin{equation}
\label{eqn:freq-cn}
\omega_{cn} = \sum_{i=i_{max} - N/2}^{i=i_{max} + N/2} PSD(\omega_i)
\end{equation}

\todo{error analysis}

\subsection{Analytical Form Fit}
The full analytical form of the idea FID PSD peak is trickier to implement

\todo{maybe cut this section}

\subsection{Lorentzian Peak}
Peak fitting routines can improve precision upon the centroid technique.  Peak fitting works well using a Lorentzian Distribution, equation \ref{eqn:lorentzian}, in the domain around the maximum frequency bin.  The Lorentzian is an approximation of the analytical solution.

\begin{equation}
\label{eqn:lorentzian}
F(\omega) = \frac{1}{\pi} \\
\frac{\Gamma / 2}{(\omega - \omega_0)^2 + (\Gamma / 2)^2}
\end{equation}

\subsection{Polynomial Phase Fit}
The most intricate frequency determination technique is involves calculating the complementary phase of the original signal.  Given the knowledge that the original signal is harmonic \note{need to validate this}, the Hilbert Transform (Eqn. \ref{eqn:hilbert-transform}) gives the imaginary phase of a signal.

\begin{equation}
\label{eqn:hilbert-transform}
H(t) = IFFT[-i \cdot \mathrm{sgn}(\omega) \cdot FFT(F(t))]
\end{equation}

\begin{equation}
\label{eqn:fid-phase}
\phi(t) = \arctan(H(t) / F(t))
\end{equation}

With an equation defined for the phase propagation of the signal, a polynomial fit is effective at extracting the frequency of the FID, $\frac{\partial \phi}{\partial t}|_{t=0}$.

\begin{equation}
\label{eqn:freq-ph}
\omega_{ph} = a_1 = \argmin_{a_0, a_1, a_2, a_3} \;
|\phi - (a_0 + a_1 t + a_2 t^2 + a_3 t^3)|
\end{equation}

With the phase as a function of time, the FID frequency manifests as the linear slope of the data.  In fact, the phase information yields frequency as a function of time.  The frequency of interest is the frequency immediately after the $\pi/2$ pulse.

\todo{error analysis}

\subsection{Sinusoid Fit}
Normalizing the signal into a sinusoid is another approach for frequency extraction utilizing the hilbert transform.  The envelope is calculable as the norm of the origina signal and the hilbert transform

\begin{equation}
\label{eqn:fid-envelope}
F_{env} = \sqrt{|F(t)|^2 + |H(t)|^2}.
\end{equation}

\noindent 
With the envelope in hand, the original signal can be normalized into an amplitude one, sinusoidal signal.  The frequency is then extracted by using a minimization routine, such as least squares.

\begin{equation}
\label{eqn:freq-sn}
\omega_{sn} = \argmin_\omega \; \\
|F(t) / F_{env}(t) - \sin(\omega t - \phi_0))|
\end{equation}

\todo{uncertainty analysis}

\section{Frequency Extraction Efficacy}

Each FID frequency extraction technique needs to be tested and validated.  A whole slew of data options can be used: ideal functional FIDs, simulated FID, E821 waveforms, E821 probes in a local \uw magnetic.  Each dataset presents a test bed for determining the precision, accuracy and failure modes of each frequency extraction technique.

\subsection{Ideal FIDs}
The simplest case, the frequency is static in time and all techniques should perform well.  If the technique cannot handle the simplest FIDs, then it will not be worth using to analyze the E989 data.  The effectiveness was testing on 10000 FIDs using a seed frequency of 47 kHz, near the expected nominal pNMR frequency of 50 kHz.  

\begin{figure}
    \label{fig:fid-ideal-freq-extraction}
    \includegraphics[width=0.45\linewidth]{fig/ideal-fid-cn}
    \includegraphics[width=0.45\linewidth]{fig/ideal-fid-lz}

    \includegraphics[width=0.45\linewidth]{fig/ideal-fid-ph1}
    \includegraphics[width=0.45\linewidth]{fig/ideal-fid-sn}
    \caption{Centroid Method, Lorentzian Method, Linear Phase Method and Sinusoid Method. \note{verify results}}
\end{figure}

\subsection{Simulated FIDs}
Integrating the Bloch Equations is a clear path to simulate FID waveforms.  The waveforms are similar to those to the ideal FIDs with some possible effects from a deviation from non-ideal, finite $\pi/2$ pulses frequency from the Larmor frequency.  The simulation uses a lower Larmor frequency of around \SI{1}{\MHz} as opposed to the real Larmor frequency of \SI{61.79}{\MHz}.  The reason being that it is more efficient to simulate at lower frequencies, because the integration step can be much larger.  Additionally, the entire system runs through a lowpass filter at \SI{200}{\kHz}, so all the higher frequency content is culled from the final FID signal.

\begin{figure}
    \label{fig:fid-sim-freq-extraction}
    \includegraphics[width=0.45\linewidth]{fig/sim-fid-cn}
    \includegraphics[width=0.45\linewidth]{fig/sim-fid-lz}

    \includegraphics[width=0.45\linewidth]{fig/sim-fid-zc}
    \includegraphics[width=0.45\linewidth]{fig/sim-fid-ph1}
    \caption{Centroid Method, Lorentzian Method, Linear Phase Method and Sinusoid Method}
\end{figure}

\subsection{Simulated Gradient FIDs}
An expected problem complication with actual magnetic field measurements arises from the presence of gradients in the field.  Any real magnetic field will have gradients over the volume of the probe which can distort the signal of the probe.  These effects can be mimicked using a superposition of simulated FIDs over a small range.  One notices node-like behavior in the envelope when gradients are present (and E821 waveforms exhibited this behavior).  The simulations aim to tests the effects of small gradients on the FID frequency extraction precision and accuracy.

The range of magnetic field gradient over the pNMR probes anticipated in the \gmtwo storage field are on the order of \SIrange{10}{100}{ppb} \note{check if ppb or ppm}. A collection of simulated FIDs was created using the Bloch Equation integrator with a range chosen to be \SI{47}{\kHz} \SI{\pm 1000}{ppb}. To implement a gradient, FIDs from the simulation collection were put into a weighted sum in a way that did not change the average frequency just the variation of over the probe.

\begin{figure}
\label{fig:sim-gradient-all-fids}
\includegraphics[width=0.9\linewidth]{fig/sim-gradient-all-fids}
\caption{A collection of FID waveforms simulated by integrating the Bloch Equations.  The step size between frequencies was chosen to be \SI{0.1}{ppb}, so that the variation between FIDs was smooth and a sum of waveforms could be used to approximate an integral.}
\end{figure}

\begin{figure}
\label{fig:fid-sim-grad}
\includegraphics[width=0.9\linewidth]{fig/fid-sim-grad-none}
\includegraphics[width=0.9\linewidth]{fig/fid-sim-grad-lin-1000ppb}
\includegraphics[width=0.9\linewidth]{fig/fid-sim-grad-quad-1000ppb}
\caption{Example simulation waveforms made using a gradient superpositions.  The top waveform has no applied gradient.  The middle has a linear gradient of \SI{1000}{ppb}.  Note that the beat frequencies emplace near complete nodes onto the waveform envelope.  The bottom waveform has a quadratic gradient of \SI{1000}{ppb}.  Note the softer waveform distortions.}
\end{figure}

The resulting frequency extraction from the gradient FIDs are presented in figures \ref{fig:fid-sim-grad-lin-results} \& \ref{fig:fid-sim-grad-quad-results}.  In the linear case, the phase fit is essentially unperturbed while the zero crossing method gains uncertainty and a small bias over the \SI{1000}{ppb} gradient applied.  The effect in the expected gradient range of \SI{\sim100}{ppb} is neglible.  In the quadratic case, much more dramatic deviations from the true frequency are present, but again the effects are small in the expected FID gradient domain.  The effect at \SI{200}{ppb} in all cases is less than \SI{10}{ppb}. \note{is this actually okay?}

\begin{figure}
\label{fig:fid-sim-grad-lin-results}
\includegraphics[width=0.45\linewidth]{fig/fid-sim-grad-lin-zc}
\includegraphics[width=0.45\linewidth]{fig/fid-sim-grad-lin-ph1}
\caption{The plot depicts the deviation from the true input frequency in linear gradient FIDs.  The deviations are negligible for the phase fit method, and small for the zero crossing method but not negligible at 1000ppb.} 
\end{figure}

\begin{figure}
\label{fig:fid-sim-grad-quad-results}
\includegraphics[width=0.45\linewidth]{fig/fid-sim-grad-quad-zc}
\includegraphics[width=0.45\linewidth]{fig/fid-sim-grad-quad-ph1}
\caption{The plots show the the deviation from the true input frequency for in quadratic gradient FIDs.  The deviations are much larger than the linear case, but still negligible at the expected gradients of a few hundred ppb.}
\end{figure}

\subsection{FID Measurements}
A stability dataset of waveforms from the field shimming per was also available to cut the FID algorithms' teeth on.  Testing against this dataset had a real difference in that the true precession frequency was not known.  One can still build distributions to get a sense for the precision and stability of a analysis method.

Figures \label{fig:fid-real-trend} \& \label{fig:fid-real-freq} show the analysis results for real FIDs.  Zero crossing and phase appear similarly effective in the context of real data.  Both methods have a precision of \SI{\sim 10}{ppb} in the study.  The phase fit central value has a shift of \SI{\sim 15}{ppb}.  \note{need to investigate this more}

\begin{figure}
\label{fig:fid-real-trend}
\includegraphics[width=\linewidth]{fig/fid-real-trend}
\caption{The detrended plot was produced by averaging all 25 probes in the shimming platform.  Then, subtracting off the average from the central probe value.  Lastly, fitting the difference to a quadratic and subtracting off the quadtratic trend.  The result shows a small difference between the zero crossing and phase fit methods, but the distributions are within agreement.}
\end{figure}

\begin{figure}
\label{fig:fid-real-freq}
\includegraphics[width=0.45\linewidth]{fig/fid-real-zc}
\includegraphics[width=0.45\linewidth]{fig/fid-real-ph1}
\caption{The histogrammed results of two different frequency extraction methods after detrending the signal.  The two distributions nearly agree with a small shift in the central value.  Interestingly, the phase fit method and zero crossign show similar precision on real data.}
\end{figure}

\section{FID Frequency Results}

\begin{table}[h]
\label{tab:fid-analysis-summary}
\caption{FID Analysis Summary}
\centering
\begin{tabular}{l c c c c c c}
    \hline
    \multicolumn{1}{c}{Data Type} & Zero Crossing & Phase Fit \\
    \hline
    Ideal                & na & na \\
    Simulated            & na & na \\
    Linear Gradient      & na & na \\
    Quadratic Gradient   & na & na \\
    Real                 & na & na \\
    \hline
\end{tabular}
\end{table}

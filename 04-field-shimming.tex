\chapter {Shimming the Magnetic Field}

\section{Hardware}

\subsection{The Storage Ring}

The \gmtwo Storage Ring was designed with a lofty goal of 1 ppm peak-to-peak magnetic field uniformity in the muon storage volume.  The pole surface was crafted with some of the purest steel and highest precision flatness that was available at the time. The purity of the steel improves uniformity by normalizing the magnetic saturation effects across the pole surface.  The flatness uniformity directly improves the gap uniformity which is a primary order proxy for the magnetic field strength between the two pole surfaces. The pole pieces have adjustable mounts called "pole feet" which offer leverage on reshaping the pole surface by small amounts. 

\todo{add image of poles+gap}

The storage ring was also engineered with several built-in knobs to the tune the field locally in azimuth.  The storage ring includes 864 wedge shims, i.e. 12 per pole piece, which occupy the space between the pole pieces and the main yoke.  They have an angled design in order to have an effect on both the normal quadrupole and dipole field.  Each pole piece has two steel runners on the outside of the uniform, flat surface.  With top and bottom sets of so-called edge shims, the design imparted a complex handle on higher order, more than quadrupole, multipoles. The last and simplest shimming knob are the steel plates dubbed the top hats.  The top hats run in 15 degree segments, two per yoke, top and bottom.  Adjusting the top hats can induce a large effect on the average field in its respective region.  

\todo{add images of shims}

After the Big Move, the storage ring was reassembled with some care to bring it into a state similar to the final E821 state of the ring with hopes that the field shimming time could be reduced.  The magnetic field produced when the ring was repowered at Fermilab was similar to the initial field at Brookhaven, so the reassembly was unfortunately not a big leg up.  The field team found themselves at a fresh start.

\subsection{Measurement Devices}

\subsubsection{Shimming Cart}
The main tool used in the rough shimming process was a platform full of equipment that fit between the pole surfaces all around the ring.  The shimming cart measured the magnetic field, the pole gap and the local temperature.  The magnetic field was measured through a matrix of 25 pNMR probes secured in a frame which defines an azimuthal plane of the size of the cross-section of the muon storage volume. The cart also possesses four capacitive distance sensors which measure the pole gap at the inner and outer radius.  A temperature probe was placed on the cart near the pNMR probes.  All materials were chosen to make as small of a magnetic perturbation as possible.  The cart traversed the ring with the help of a stepper motor and flexible cabling to impart steps forward.  At each location the probes and capacitec made measurements, then the cart moved on.  A full scan of the ring with the standard step size took about 3 hours.

\todo{engineering diagram of cart}

\subsection{Laser Tracker}
The shimming cart was equipped with four reflective corner cube mirrors which could be tracked with a laser tracker.  A commercial, API laser tracker was installed in the center of the ring and used to establish the position of the measurements taken by the shimming cart.  The most important value from the laser tracker was the azimuthal angle, $\phi$, but the device also reported height, $z$, and radius, $r$.  The laser tracker measured with precision, $\delta\phi \approx \SI{0.1}{\deg}$, $\delta r \approx \SI{0.1}{\milli\meter}$, and $\delta z \approx \SI{25}{\micro\meter}$ \note{verify these}.

\subsection{Tilt Sensor}
A custom tilt sensor built by \uw was another important tool.  The tilt sensor consisted of a \SI{29}{\cm} by \SI{12}{\cm} aluminum base plate with two mounted electrolytic bubble levels.  It had three spherical feet both top and bottom to define a consistent plane.  The tilt sensors were able to read out at a precision of about \SI{2}{\micro\radian}.  The sensors take more than 30 minutes to fully settle though, so most measurements were made with lower precision to reduce measurement time.

\todo{add calibration images}

\subsection{Pole Moves}

\subsection{Edge Shim Study}

\subsection{Shim Optimization}

\subsection{Lamination Audible}

\section{Results}
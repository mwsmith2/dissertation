\chapter {Shimming the Magnetic Field}

\section{Hardware}

\subsection{The Storage Ring}

The \gmtwo Storage Ring was designed with a lofty goal of 1 ppm peak-to-peak magnetic field uniformity in the muon storage volume.  The pole surface was crafted with some of the purest steel and highest precision flatness that was available at the time. The purity of the steel improves uniformity by normalizing the magnetic saturation effects across the pole surface.  The flatness uniformity directly improves the gap uniformity which is a primary order proxy for the magnetic field strength between the two pole surfaces. The pole pieces have adjustable mounts called "pole feet" which offer leverage on reshaping the pole surface by small amounts. 

\todo{add image of poles+gap}

The storage ring was also engineered with several built-in knobs to the tune the field locally in azimuth.  The storage ring includes 864 wedge shims, i.e. 12 per pole piece, which occupy the space between the pole pieces and the main yoke.  They have an angled design in order to have an effect on both the normal quadrupole and dipole field.  Each pole piece has two steel runners on the outside of the uniform, flat surface.  With top and bottom sets of so-called edge shims, the design imparted a complex handle on higher order, more than quadrupole, multipoles. The last and simplest shimming knob are the steel plates dubbed the top hats.  The top hats run in 15 degree segments, two per yoke, top and bottom.  Adjusting the top hats can induce a large effect on the average field in its respective region.  

\todo{add images of shims}

After the Big Move, the storage ring was reassembled with some care to bring it into a state similar to the final E821 state of the ring with hopes that the field shimming time could be reduced.  The magnetic field produced when the ring was repowered at Fermilab was similar to the initial field at Brookhaven, so the reassembly was unfortunately not a big leg up.  The field team found themselves at a fresh start.

\subsection{Measurement Devices}

\subsubsection{Shimming Cart}
The main tool used in the rough shimming process was a platform full of equipment that fit between the pole surfaces all around the ring.  The shimming cart measured the magnetic field, the pole gap and the local temperature.  The magnetic field was measured through a matrix of 25 pNMR probes secured in a frame which defines an azimuthal plane of the size of the cross-section of the muon storage volume. The cart also possesses four capacitive distance sensors which measure the pole gap at the inner and outer radius.  A temperature probe was placed on the cart near the pNMR probes.  All materials were chosen to make as small of a magnetic perturbation as possible.  The cart traversed the ring with the help of a stepper motor and flexible cabling to impart steps forward.  At each location the probes and capacitec made measurements, then the cart moved on.  A full scan of the ring with the standard step size took about 3 hours.

\todo{engineering diagram of cart}

\subsection{Laser Tracker}
The shimming cart was equipped with four reflective corner cube mirrors which could be tracked with a laser tracker.  A commercial, API laser tracker was installed in the center of the ring and used to establish the position of the measurements taken by the shimming cart.  The most important value from the laser tracker was the azimuthal angle, $\phi$, but the device also reported height, $z$, and radius, $r$.  The laser tracker measured with precision, $\delta\phi \approx \SI{0.1}{\deg}$, $\delta r \approx \SI{0.1}{\milli\meter}$, and $\delta z \approx \SI{25}{\micro\meter}$ \note{verify these}.

\subsection{Tilt Sensor}
A custom tilt sensor built by \uw was another important tool.  The tilt sensor consisted of a \SI{29}{\cm} by \SI{12}{\cm} aluminum base plate with two mounted electrolytic bubble levels.  It had three spherical feet both top and bottom to define a consistent plane.  The tilt sensors were able to read out at a precision of about \SI{2}{\micro\radian}.  The sensors take more than 30 minutes to fully settle though, so most measurements were made with lower precision to reduce measurement time.

\todo{add calibration images}
\section{Adjustments}

Though it was clear from the start that there was much work to be done in order to shim the field, the path forward was not as obvious.  After gauging the measurement devices and subsequently the state of the storage field, it was decided that several stages of adjustments were necessary.  The first stage was leveling the ring with respect to gravity.  The second was a full iteration of adjusting the feet on each pole piece.  The third was a survey of whether or not high order multipole adjustments via edge shims were needed.  The fourth was optimization of the field using the built-in shimming kit.  And, the final round of adjustments was thhe implementation of a surface of measured iron segments which really pushed us into new territory from the E821 experiment.

\subsection{Ring Leveling}
At an intermediate point in the shimming process, it became clear that the plane of the storage ring and the plane defined by gravity were not aligned.  Furthermore measurements indicated that that plane of the ring was different from the plane that had been measured upon construction of the ring.  It would seem that the floor had settled a millimeter or two.  In addition to correcting the ring alignment, the adjustment procedure also afforded an opportunity to see the effect that future floot settling might have on the magnetic field.

The laser data clearly shows the tilt plane of storage ring (see figure \ref{fig:ring-leveling-plan}).  The deviations of the ring plane were about \SI{1}{\milli\meter} from the average value.  The tilt plane was also visible in the radial tilts of the poles around the ring.  An adjustment plan was crafted to fix the height and radial tilts of each yoke.  The plan to to level the ring used only adjustments on the legs of each yoke. With yoke leg adjustments the interface between each yoke remains fixed, so the adjustment plan locked yokes together on the boundaries.  The plan used linear fits of the laser in small angular windows on each side of the yoke boundaries to determine the target height for the inside of the yoke.  The radial tilt measurements were used to calculate differential adjustments between the inner and outer legs of a yoke while of course minding the fixed interface. 

\begin{figure}
\includegraphics[width=0.9\linewidth]{fig/ring-leveling-plan.png}
\caption{The ring plane had a rather large tilt plane built into it.  The tilt plane was removed using only adjustments on the yoke legs.  The adjustments were devised both bring the ring plane close to the plane defined by gravity and rein in yokes with wild radial tilts.}
\label{fig:ring-leveling-plan}
\end{figure}

A team of mechanical engineers led by Eric Voirin performed the yoke adjustments.  Adjustments were made incrementally in steps of approximately \SI{100}{\micro\meter}, the reason being an effort to minimize differential stress between the yokes.  During the process, the shimming laser tracker was replaced with the alignment team Hamar laser.  Progress was measured periodically using a floating corner cube placed at three specific pole positions around the ring.  The measurements with a sine fit overlay are depicted in figure \ref{fig:ring-leveling-steps}.  They give a strong sense of the success of the procedure.  The whole process took two days to affect the initial plan and one more day of minor adjustments.

\begin{figure}
\includegraphics[width=0.9\linewidth]{fig/ring-leveling-steps.png}
\caption{The image depicts iterations of yoke leg adjustments made to level the storage ring.  Three locations around the ring were measured using the Hamar laser system.  Those locations are represented by the points in the plot.  The points were fit to sine wait to illustrate how the leveling effects in the unmeasured sections of the ring.}
\label{fig:ring-leveling-steps}
\end{figure}

The ring leveling was very successful.  Yoke leg adjustments managed to remove the tilt plane from the ring and help flatten the radial tilt of the poles (\ref{fig:ring-leveling-final}.  After ring leveling the ring plane was flat to about \SI{300}{\micro\meter}.  While flattening the ring simplified general \gmtwo detector alignment and made future adjustments simpler for the field team, one of the most important results was the effect on the field.  In theory, the field should not depend on the ring plane or the relative yoke orientations, but the ring leveling provided an opportunity to symbiotically measure the effect.  Figure \ref{fig:ring-leveling-field-effect} contains a plot of the difference between the azimuthally averaged field before and after leveling.  There is a small change in the average dipole field which is hard to ascribe a cause with certainty (field drift, adjusted hysteresis from moving yokes, etc.).  The higher order symmetries remain essentially unchanged though, and that is the important result.

\begin{figure}
\includegraphics[width=0.9\linewidth]{fig/ring-leveling-final.png}
\caption{The image shows the original ring plane in red and the final ring plane in green.  The plot speaks for itself as to the success of leveling the plane of the ring.}
\label{fig:ring-leveling-final}
\end{figure}

\begin{figure}
\includegraphics[width=0.9\linewidth]{fig/ring-leveling-field-effects.png}
\caption{Another important test from the ring leveling experience is the field effects.  The plots show the magnetic field averaged over the azimuth before and after ring leveling.  The takeaway is a small possible effect on the average field (though this could also be drift and other effects), and virtually no effect on higher order multiples.  If necessary, the ring could be leveled again without worry of destroying the field uniformity.}
\label{fig:ring-leveling-field-effects}
\end{figure}

\subsection{Pole Moves}

Adjusting the orientation and shape of the pole pieces was the first substantial stage in optimizing the uniformity of \gmtwo storage field.  Having previously established calibrations for the field effects,  the field team tested the model on some of the most aberrant poles with some success.  It quickly became clear that we could not make planned adjustments on a pole-by-pole basis though.  The cost of adjusting a single pole was fairly large taking hours to crane out, replace shims in the pole feet, and crane back in, so the plan needed to do as much as possible with a single pass of pole adjustments.  What was needed was a global model of the pole geometry.

\subsubsection{The Bottom Pole Model}

Building an accurate global model of the pole surfaces was not a simple task.  We had four possible measurements to use: the laser tracker, the tilt sensor, the capacitec sensors, and the field measurements.  The team elected to ignore field data as it was difficult to decouple the effect of current shim positions, impurities, and perturbations from hardware that broke the azimuthal symmetry.  In the right combination, the sensor data is just enough to build a complete model of the pole surfaces.  Let's build the model from the bottom poles up.

The bottom poles were characterized using a synergistic combination of tilt sensor data and laser measurements.  The shimming cart rode on the bottom poles along the edge shims at the inner radius of the pole \note{reference shimming cart figure}. The height value reported by the laser then directly represented the the inner radius of the bottom pole surface.  From the inner radius of the bottom pole, the outer radius can be extrapolated from tilt sensor measurements.  Each pole was broadly characterized with a radial and azimuthal tilt measurement taken from the center defining a rough plane for the pole.  The pole interfaces were characterized with a set of three tilt measurements near the gap: one on the upstream pole, one straddling both poles, and one on the downstream pole.  The pole step measurement is very sensitive to average elevation changes and rotation mismatches between poles.  The pole model folds the laser height data in with a radial tilt model that smoothly varies across the pole, and then, restricts the pole interfaces with the pole step tilt measurements.  An entire global plan for the bottom pole moves was made from this model, see figure \ref{fig:pole-moves-bottom-plan}

\begin{figure}
\includegraphics[width=0.9\linewidth]{fig/pole-moves-bottom-plan}
\caption{The plot visually represents the plan for adjusting pole feet using circles for the inner feet, triangles for the outer feet against their azimuth location.  The laser data which anchored the pole surface model is also plotted in scatter form.  Notice that the inner feet changes are minor adjustments on the laser data, and the outer feet changes fold in the radial tilt measurements on the poles.}
\label{fig:bottom-pole-adjustment-plan}
\end{figure}

\subsubsection{The Top Pole Model}

The model for the top pole surfaces had to be constructed on the foundation of the bottom pole model.  While this situation was not ideal, it did work out.  Starting from the model for the bottom poles, the sum of the two inner capacitec measurements acted as a proxy for the gap between the upper and lower poles, so the sum of the bottom pole model inner radius and the two inner capacitec values represented the inner band of the top pole model.  With an established value for the inner band of the top poles, the same extrapolation to the outer band and pole interfaces was employed as with the bottom poles.  The top and bottom outer bands could then be used to valid the pole model overall by comparing the predicted gap in the outer bands to the gap measured by the outer capacitec sensors \todo{add validation plot}.  The model was of course not perfect, but ready for a full sweep of pole moves, top and bottom.

\subsubsection{First Round}

Implementing pole adjustments proved to be an involved procedure.  A typical day went as follows: John Najdzion would come in early around 6am and carefully transport 3 or 4 poles onto work blocks, the field team would come with micrometers and assorted shimstock to implemented a prescription of changes to 1/4 mil (\SI{6}{\micro\meter}), and the John Najdzion \note{spelling} would reseat the poles into the ring.  The first full round of pole movements took over a month to implement.  The results speak for themselves though.

\begin{figure}
\includegraphics[width=0.9\linewidth]{fig/pole-moves-bottom-tilts}
\caption{The radial tilts in red and the azimuthal tilts in green of the bottom poles are shown before pole moves in lighter color and after in darkened color.  The dotted horizontal bands shown represent the target range for the tilt uniformity.  The improvement from pole movements is clear.}
\label{fig:pole-moves-bottom-tilts}
\end{figure}

\subsubsection{Round Two}

The model was not perfect and nor was the predictability of adjusting pole feet.  The results from the first pass at pole adjustments were good, but a second round of tweaks really completed the picture.  The primary goal of the second pass at pole adjustments was to elminate large extant deviations and close remaining pole steps in the center of each pole interface to within \SI{0.5}{mil} (\SI{12.5}{\micro\meter}.  Many of the adjustments were done without removal of the poles, since the front pole feet were accessible after untorquing the super bolts and propping the weight of the bottom poles up on a jackstand of sorts.  The jackstand was an adjustable pole foot prototype developed duing the E821 era.  The top poles did not require a jackstand, since gravity assisted displacing the poles from the yoke in that case.  In each specific case, the field team weighed the value of retrying a full pole adjustment against small tweaks on the inner feet.  

\subsubsection{Final Results}

While still making pole adjustments, the field team began to implement shim adjustments, so the entirety of the field improvements show in figure \ref{fig:pole-moves-field-comparison} is not due to the pole moves.  The main improvements from the pole moves is the elminations of sharp spikes in the average field that were caused by large elevation changes in the pole surface due to pole interface mismatches, along with the elimination of the the average normal quadruopole moment by adjusting the radial tilts top and bottom.

\begin{figure}
\includegraphics[width=0.9\linewidth]{fig/pole-moves-field-comparison}
\caption{The azimuthally average field plots are shown with field prior to pole moves on the left and field after on the right.  A little care is necessary in interpreting attributions for the improvements though, because some shim adjustments were made before the pole movements had completed.  The major improvement shown in the plot is the elimination of the \SI{-18}{ppm} normal quadrupole moment.  Some improvement in the average skew quadropole is also evident.}
\label{fig:pole-moves-field-comparison}
\end{figure}


\subsection{Edge Shim Study}

\todo{spend time piecing together edge shim images and story}

\subsection{Shim Optimization}

\todo{spend time piecing together shim optimization images and story}

\subsection{Lamination Audible}

\todo{spend time piecing together lamination images and story}

\section{Results}
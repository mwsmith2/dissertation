\chapter{FID Analysis}

The signal generated in the pNMR probes contains frequency information that represents the effective field seen by the probe's protonated \note{might be wrong word} volume.  A fraction of the polarized protons rotate into orthogonal plane, then precess at a frequency proportional to the dominant field that had previously polarized the particles.  The average frequency of the the precessing protons immediately after the $\pi/2$ pulse is target of analysis.  That value represents the magnetic field in the volume.  The digitize precession signal recorded as the protons slowly rotate back into the original polarization and decohere is called the Free Induction Decay (FID).

\todo{insert equation for macro frequency of protons}

\subsection{Free Induction Decays}

The polarized protons involved in the FID evolve according to the Bloch Equations.  An ordinary differential equation that couples magnetization and field in different dimensions.

\begin{equation}
\frac{dM_x(t)}{dt} = 
\gamma (\mathbf{M}(t) \times \mathbf{B}(t))_x - \frac{M_x(t)}{T_2}
\label{eqn:bloch-x}
\end{equation}

\begin{equation}
\frac{dM_y(t)}{dt} = 
\gamma (\mathbf{M}(t)\times \mathbf{B}(t))_y - \frac{M_y(t)}{T_2}
\label{eqn:bloch-y}
\end{equation}

\begin{equation}
\frac{dM_z(t)}{dt} = 
\gamma (\mathbf{M}(t) \times \mathbf{B}(t))_z - \frac{M_z(t) - M_0}{T_2}
\label{eqn:bloch-z}
\end{equation}

The ideal FID waveform from a pure field value and perfect $\pi/2$ pulse can be solved exactly from the differential equations.  The resulting equation, \ref{eqn:ideal-fid}, exhibits the primary characteristics of an FID, a sinusoid term representing the proton precession and an exponential decay envelope representing the $T_2$ relaxation.  Figure \ref{fig:ideal-fid} exhibits the waveform of an ideal FID.

\begin{equation}
FID(t) = e^{-t/T_2} \sin(\omega t - \phi_0)
\label{eqn:ideal-fid}
\end{equation}

\begin{figure}
\todo{insert ideal FID figure}
\label{fig:ideal-fid}
\end{figure}

In a realistic situation, the FID signal is a summation over varying field values, and the signal bears strong deviations from the ideal FID form.  Nevertheless, the ideal signal is a valid testing bed for frequency extraction algorithms.

\subsection{Frequency Extraction Methods}

As in most data analysis, the first step is cleaning and characterizing the data.  In the case of FIDs, the process involves removing the baseline \note{and possibly trend} and determining the start time and stop time of useful FID signal within the frame of the entire recorded waveform.  The baseline value is calculated using initial segment of the waveform.  The signal range is computed using the maximum amplitude around the signal baseline and applying a threshold on to find the first and last sections of the waveform with an envelope above that threshold.

\subsubsection{Zero Crossing}
The simplest method for determining the FID frequency involves counting zero crossings of the signal.  
\todo{error analysis}

\subsubsection{Spectral Centroid}
Another straightforward technique employed to extract the FID frequency relies on the spectral density of the signal. 

\begin{equation}
FFT(\omega_k) = \sum_{n=0}^{n=N} e^\frac{-i \omega_0}{2\pi k} f(x_n)
\label{eqn:fid-fft}
\end{equation}

\begin{equation}
PSD(\omega_k) = |FFT(\omega_k)|^2
\label{eqn:fid-psd}
\end{equation}

In the power spectral density (PSD), the FID frequency manifests as a peak energy.  The frequency is computed as a frequency weighted  sum of bins symmetrically around the peak value.

\begin{equation}
\omega_{CN} = \sum_{i=i_{max} - N/2}^{i=i_{max} + N/2} PSD(\omega_i)
\label{eqn:freq-cn}
\end{equation}

\todo{error analysis}

\subsubsection{Lorentzian Peak}
Peak fitting routines can improve precision upon the centroid technique.  Peak fitting works well using a Lorentzian Distribution, Eqn. \ref{eqn:lorentzian} around the maximum frequency bin.

\begin{equation}
F(\omega) = \frac{1}{\pi}\frac{\frac{1}{2} \Gamma}{(\omega - \omega_0)^2 + (\frac{1}{2} \Gamma)^2}
\label{eqn:lorentzian}
\end{equation}

\subsubsection{Analytical Form Fit}
The full analytical form of the idea FID PSD peak is trickier to implement

\todo{maybe cut this section}

\subsubsection{Polynomial Phase Fit}
The most intricate frequency determination technique is involves calculating the complementary phase of the original signal.  Given the knowledge that the original signal is harmonic \note{need to validate this}, the Hilbert Transform (Eqn. \ref{eqn:hilbert}) gives the imaginary phase of a signal.

\begin{equation}
H(t) = IFFT(-i \mathrm{sgn}(\omega) \cdot FFT(F(t)))
\label{eqn:hilbert}
\end{equation}

\begin{equation}
\phi(t) = \arctan(H(t) / F(t))
\label{eqn:phase}
\end{equation}

With the phase as a function of time, the FID frequency manifests as the linear slope of the data.  In fact, the phase information yields frequency as a function of time.  The frequency of interest is the frequency immediately after the $\pi/2$ pulse.

\todo{error analysis}

\subsection{Frequency Extraction Efficacy}

Each FID frequency extraction technique needs to be tested and validated.  A whole slew of data options can be used: ideal functional FIDs, simulated FID, E821 waveforms, E821 probes in a local \uw magnetic.  Each dataset presents a test bed for determining the precision, accuracy and failure modes of each frequency extraction technique.

\subsubsection{Ideal FIDs}
The simplest case, the frequency is static in time and all techniques should perform well.  If the technique cannot handle the simplest FIDs, then it will not be worth using to analyze the E989 data.  The effectiveness was testing on 10000 FIDs using a seed frequency of 47 kHz, near the expected nominal pNMR frequency of 50 kHz.  

\begin{figure}
    \label{fig:ideal-fid-spectral}
    \includegraphics[width=0.45\linewidth]{fig/ideal-fid-cn}
    \includegraphics[width=0.45\linewidth]{fig/ideal-fid-lz}
    \caption{Centroid Method and Lorentzian Method}
\end{figure}

\begin{figure}
    \label{fig:ideal-fid-time-domain}
    \includegraphics[width=0.45\linewidth]{fig/ideal-fid-ph1}
    \includegraphics[width=0.45\linewidth]{fig/ideal-fid-sn}
    \caption{Linear Phase Method and Sinusoid Method}
\end{figure}

\note{verify results, combine results into single image}

\subsubsection{Simulated FIDs}
Integrating the Bloch Equations is a clear path to simulate FID waveforms.  The waveforms are similar to those to the ideal FIDs with some possible effects from a deviation from non-ideal, finite $\pi/2$ pulses frequency from the Larmor frequency.  The simulation uses a lower Larmor frequency of around \SI{1}{\MHz} as opposed to the real Larmor frequency of \SI{61.79}{\MHz}.  The reason being that it is more efficient to simulate at lower frequencies, because the integration step can be much larger.  Additionally, the entire system runs through a lowpass filter at \SI{200}{\kHz}, so all the higher frequency content is culled from the final FID signal.

\begin{figure}
    \label{fig:sim-fid-spectral}
    \includegraphics[width=0.45\linewidth]{fig/sim-fid-cn}
    \includegraphics[width=0.45\linewidth]{fig/sim-fid-lz}
    \caption{Centroid Method and Lorentzian Method}
\end{figure}

\begin{figure}
    \label{fig:sim-fid-time-domain}
    \includegraphics[width=0.45\linewidth]{fig/sim-fid-zc}
    \includegraphics[width=0.45\linewidth]{fig/sim-fid-ph1}
    \caption{Linear Phase Method and Sinusoid Method}
\end{figure}

\todo{summary table}

\subsubsection{Simulated Gradient FIDs}
An expected problem complication with actual magnetic field measurements arises from the presence of gradients in the field.  Any real magnetic field will have gradients over the volume of the probe which can distort the signal of the probe.  These effects can be mimicked using a superposition of simulated FIDs over a small range.  One notices node-like behavior in the envelope when gradients are present (and E821 waveforms exhibited this behavior).  The simulations aim to tests the effects of small gradients on the FID frequency extraction precision and accuracy.

The range of magnetic field gradient over the pNMR probes anticipated in the \gmtwo storage field are on the order of SIrange{10}{100}{ppb} \note{check if ppb or ppm}. A collection of simulated FIDs was created using the Bloch Equation integrator with a range chosen to be \SI{47}{\kHz} \SI{\pm 1000}{ppb}. To implement a gradient, FIDs from the simulation collection were put into a weighted sum in a way that did not change the average frequency just the variation of over the probe.

\begin{figure}
\label{fig:sim-gradient-all-fids}
\includegraphics[width=0.9\linewidth]{fig/sim-gradient-all-fids}
\caption{A collection of FID waveforms simulated by integrating the Bloch Equations.  The step size between frequencies was chosen to be \SI{0.1}{ppb}, so that the variation between FIDs was smooth and a sum of waveforms could be used to approximate an integral.}
\end{figure}

\begin{figure}
\label{fig:sim-gradient-lin-grad}
\includegraphics[width=0.9\linewidth]{fig/sim-gradient-lin-grad-0ppm}
\includegraphics[width=0.9\linewidth]{fig/sim-gradient-lin-grad-100ppm}
\includegraphics[width=0.9\linewidth]{fig/sim-gradient-lin-grad-200ppm}
\caption{Example simulation waveforms made using a linear gradient sum over frequencies.  The beat frequencies emplace nearly total nodes onto the waveform envelope.}
\end{figure}

\begin{figure}
\label{fig:sim-gradient-quad-grad}
\includegraphics[width=0.9\linewidth]{fig/sim-gradient-quad-grad-0ppm}
\includegraphics[width=0.9\linewidth]{fig/sim-gradient-quad-grad-100ppm}
\includegraphics[width=0.9\linewidth]{fig/sim-gradient-quad-grad-200ppm}
\caption{Example simulation waveforms made using a quadratic gradient sum over frequencies.  The beat frequencies impart softer waists onto the waveform envelope compared to the purely linear gradient.}
\end{figure}

\todo{redo gradient analysis effects.}

\subsubsection{FID Measurements}
A small set of fully digitized waveforms from E821 was also available to cut the FID algorithms' teeth on.  Testing against this dataset had a real difference in that the true precession frequency was not known.  One can still build distributions to get a sense for the precision and stability of a analysis method.

\todo{dig up or redo}

\subsection{FID Frequency Results}

\todo{really depends on results that I insert above}

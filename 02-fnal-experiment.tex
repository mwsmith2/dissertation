% Chapter details the methods and hardware used to create
% and measure the g-2 Magnetic Field.
\chapter {E989: The Muon \gmtwo Experiment} \label{ch:expt}

\section{The Big Picture} \label{sec:expt-big-picture}

\subsection{Final Deliverable}

The principal result from the muon \gmtwo experiment can be represented in a single expression in which there are two quantities measured very precisely by the experiment.

\begin{equation}
\label{eqn:g-2-results}
a_\mu = \frac{\omega_a}{\omega_p} \frac{\mu_p}{\mu_e} \\
\frac{m_\mu}{m_e} \frac{g_e}{2}
\end{equation}

\noindent
In equation \ref{eqn:g-2-results}, the two frequencies in the first fraction are measured in the experiment.  The spin precession frequency, $\omega_a$, is extracted with data from the calorimetry system described in more detail in section \ref{sec:spin-precession}.  The other frequency is represents the magnetic field strength as measured by the pulsed nuclear magnetic resonance (pNMR) magnetometer system from the experiment which is discussed further in section \ref{sec:magnetic-field}.  All other values in equation \ref{eqn:g-2-results} have already been measured in other experiments. The CODATA values \cite{codata} for the magnetic moment and mass ratio and reference \cite{g-e-measurement} for the g-factor of the electron are given below.

\begin{align}
\mu_p/\mu_e & = -658.210\;6848(54) \\
m_\mu/m_e & = 206.768\;2843(52) \\
g_e/2 & = 1.001\;159\;652\;180\;73(28)
\end{align}

Another common expression of the \mugmtwo results derives from the frequencies involved in the experiment.

\begin{equation}
\label{eqn:g-2-results-2}
\frac{\mathcal{R}}{\lambda^+ - \mathcal{R}}
\end{equation}

$\mathcal{R}$ is the ratio of $\omega_a$, the spin precession frequency, and $<\omega_p>$, the muon weighted magnetic field measurement which are both measured by the experiment. And, $\lambda^+$ is the ratio of the magnetic moment of the muon to that of the proton which is measured in muonium a hyperfine splitting experiment \cite{muonium-hyperfine}.  Using the CODATA values \cite{codata}, the ratio computes to

\begin{equation}
\label{eqn:muon-to-proton-mu-ratio}
\lambda^+ = \mu_{\mu^+} / \mu_p = 3.183\;345\;139(10).
\end{equation}

\subsection{Uncertainty Goals}

The precision goals for the measurement round out the essential of the experimental.  The overall, final precision goal on $a_\mu$ is \SI{140}{ppb}.  The statistical uncertainty in the error budget is allotted \SI{100}{ppb} and achieved through recording $1.5\times10^{11}$ muon events.  The systematics limit on $\omega_a$ is an uncertainty of \SI{70}{ppb}.  And, the systematics limit on $<\omega_p>$ is allowed an uncertainty \SI{70}{ppb}.  The individual terms summed in quadrature represent the entire error budget for the muon \gmtwo experiment.  The uncertainty goal improves nearly four-fold over the precision of the previous experiment at BNL.

\subsection{Experiment Logistics}

The \mugmtwo experiment can be broken into several stages.  In the subsequent sections the following stages are discussed:

\begin{itemize}[noitemsep]
\item{Muon Production}
\item{The Storage Ring}
\item{Muon Injection}
\item{The Magnetic Field}
\item{Spin Precession}
\end{itemize}

\section{Muon Production} \label{sec:muon-production}

The first stage in an experiment that measures a property of muons is of course producing muons.  The particle beamline facilities and expertise at Fermilab has developed a new muon beam to deliver muons to the \gmtwo experiment.  Figure \ref{fig:muon-production-beamline} depicts the beamline components at Fermilab.  The full chain of muon production goes from protons to pions to muons.  Pions are created when a beam of \SI{8}{\GeV/c} protons collides with a nickel target.  The collision produces many pions which continue downstream where a few things happen: the pions are focused via an electrostatic lithium lens, extant protons are filtered out, the pions are selected within a momentum spread of \SIrange{0.02}{0.05}{\frac{\delta p}{p}}, and the pions undergo in-flight weak decay into muons \cite{e989-tdr}.

\begin{figure}
\label{fig:muon-production-beamline}
\includegraphics[width=0.9\linewidth]{fig/muon-production-beamline}
\caption{The accelerator beamline that delivers muons for E989.  Protons are accelerated to \SI{8}{\GeV/c} and bunched in the Recycler Ring.  Then the protons hit the pion production target and lithium focusing lens.  The pions propagate down the P1-P2-P3 line to the Delivery Ring where muons are bunched and delivered to MC-1, the \mugmtwo experimental hall, at \SI{12}{\second^-1} \cite{e989-tdr}. \todo{double check}}
\end{figure}

The pion beam traverses a \SI{296}{\meter} path to the storage ring \cite{e989-tdr}.  The length gives adequate time for nearly all of the pions to decay into muons.  Through the pion decay process the high and low energy muons have a net spin polarization (as discussed in section \ref{sec:muon-attributes}), so the beamline design selects for the appropriate muon energy and achieves strong polarization. With this technique, a spin polarization of around \SI{95}{\percent} can be achieved.  A bunch of muons produced in the beamline is referred to as a ``fill'', and these fills will arrive at the storage ring at an average rate of \SI{12}{\second^{-1}}\cite{e989-tdr}.

\section{The Storage Ring} \label{sec:storage-ring}

The magnetic storage ring is the hardware for creating the main magnetic field for the experiment.  The major components include three superconducting NbTi/Cu coils, a dozen ``C''-shaped flux return yokes in the form of steel blocks and plates, 72 high-purity steel poles, and a built-in shim kit with more than 1,000 tunable knobs.  The full assembly is quite impressive standing around \SI{3}{\meter} tall, \SI{15}{\meter} wide, and weighing more than 700 tons \cite{e989-tdr}.  The storage ring was designed with the objective of creating a magnetic field of \bmagic that is uniform as possible over a toroidal muon storage volume with a major radius of \rmagic and a minor radius of \SI{4.5}{\cm}.

\begin{figure}
\label{fig:magnet-cross-section}
\centering
\includegraphics[width=0.9\linewidth]{fig/magnet-cross-section}
\caption{The cross-sectional view of the storage ring.  The magnetic field is produced by currents in the three superconducting coils.  The field strength is primarily determined by the geometry for the flux capture in the ``C'' yoke and pole surfaces.  The field can be adjusted locally in azimuthal using a built-in shim kit that includes top hat shims, edge shims, and wedge shims.}
\end{figure}

The field itself is produced by the superconducting coils.  A current of $5176\;A$ is driven through all three coils.  Two of the coils reside at a smaller radius than \rmagic and one of the coils resides at a larger radius (see figure \ref{fig:magnet-cross-section}).  The outer coil contains twice as much superconducting materical as the two inner coils for balance.  The inner and outer coils run with opposing currents to create a nearly vertical B-field in the space between them.  By design, the return yoke draws in magnetic flux to increase the strength of field.  The precision machined, high purity pole pieces provide an extremely flat surface to increase the field uniformity.  To first approximation the magnetic field uniformity depends on the geometric uniformity of the gap between the two pole surfaces.  With some care to recreate the conditions of the magnet assembly at the end of E821, the storage ring was assembled at Fermilab and powered to a field with variations of around \ppm{1400} as show in figure \ref{fig:initial-field}.

\begin{figure}
\label{fig:initial-field}
\centering
\includegraphics[width=0.9\linewidth]{fig/initial-dipole-field.png}
\caption{The initial dipole (average at azimuthal point) magnetic field when the storage field was first fully measured at Fermilab.  The peak-to-peak variation was around \ppm{1400}.  The umbrella-like sub-structure occurs every 10 degrees and corresponds to the changing gap between the slightly curved pole pieces.  The horizontal bands represent the average field value and the uniformity goals set forth by the experiment.  The band formed by the two dotted lines around the average value indicates the experimental goal of \SI{\pm 25}{ppm}.}
\end{figure}

The last essential piece of the magnetic storage ring is the intrinsic shim kit.  There are over 1,000 knobs built into the ring hardware to make azimuthally localized changes to the field. The collection of edge shims, wedge shims, and top hats are designed to adjust specific symmetries in the magnetic field.  The shims are discussed in further detail in chapter \ref{ch:shimming}.

\section{Muon Injection} \label{sec:muon-injection}

Storing muons in the ring is not quite so simple as just bringing a muon beam to the ring.  The muons are stored in a strong vertical magnetic field with fringe effects reaching meters away from the storage region.  The natural path of muons through the fringe field is a non-linear tunnel which is not simple to predict or manufacture in the storage ring.  The solution to simplify injection was to create a volume of magnetic field to cancel the storage and fringe fields. The hardware is called the inflector.  The inflector is a cleverly designed magnet which is able to achieve a strong, nearly uniform vertical field over a small volume, and contain much of the flux from leaking and perturbing surrounding fields.  The flux capture was later improved by adding a superconducting shield to the outside.  The inflector was designed to be adjustable over small angles which allowed the apparatus to optimize the nearly straight injection of muons. See figure \ref{fig:expt-inflector} for an image of the inflector and the beam trajectory through the inflector. \cite{e989-tdr, e821-prd}

\begin{figure}
\label{fig:expt-inflector}
\includegraphics[width=0.9\linewidth]{fig/expt-inflector}
\caption{The inflector diagram on the left illustrates the relative position of the inflector as the bridge from essentially outside of the storage field through to within the storage volume.  The trajectory of the beamline through the inflector area is depicted in the right plot.}
\end{figure}

The second problem with injecting muons comes from orbital mismatch of muon trajectory from the downstream end of the inflector (see figure \ref{fig:expt-fast-kicker}.  The muons are injected into a nearly uniform vertical magnetic field which forces charged particles to take nearly circular trajectories in the field volume.  The problem arises when the muons complete their orbit around the storage ring and return to the same point they started at, the downstream end of the inflector.

\begin{figure}
\label{fig:expt-kicker-correction}
\centering
\includegraphics[width=0.6\linewidth]{fig/expt-kicker-correction}
\caption{The figure illustrates the orbital mismatch problem with muon injection.  The red track illustrates the problem where the muons collide with the injection point, and the blue track illustrates the good orbit achieved after a perfect kick.}
\end{figure}

Muon injection requires a fast kicker to impart an angular shift on all muons in the fill. In figure \ref{fig:expt-kicker-correction}, the red track illustrates the problem and the blue track illustrates the good orbit achieved after a perfect kick. Optimizing the design of the \gmtwo fast magnetic kicker was no trivial task.  The kicker ideally would produce a moderate magnetic field around \SI{275}{\gauss} with very sharp edges, $\mathcal{O}(10\;ns)$. The reality is messier, but the completion of a newly designed kicker will improves muon storage \cite{e989-tdr}.  

\section{Magnetic Field} \label{sec:magnetic-field}

The magnetic field is of critical importance to the muon \gmtwo experiment.  The magnetic field of \bmagic puts muons of \pmagic into cyclotron motion at a radius of \rmagic.  The prescribed parameters lock a fraction of the injected muons in cyclotron motion until they decay.  To first order the value of the magnetic field directly affects the rate of spin precession, eqn. \ref{eqn:omega-spin}, and cyclotron frequency, eqn. \ref{eqn:omega-cyclotron}.  In this light, the average magnetic field must be well measured, since it folds into the determination of \wa.  To second order the magnetic field couples to the symmetries in the muon beam influencing the beam dynamics of the stored muons.  These deviations in beam dynamics can cause the muons to experience a field that does not represent the average field and therefore alter the determination of the expectation value for \wa.

\subsection{The Field Expansion}

The ideal magnetic field for the experiment is perfectly vertical at \bmagic with no deviations.  However, the Maxwell equations do allow for such a perfect field within finite space.  Reality requires that the experiment consider the quality of the central field and the effects of perturbations to the field.  First, let's define the expansion to develop a common vocabulary for speaking about field perturbations, eqn. \ref{eqn:field-expansion}.  The expansion assumes the domain where $B_r \ll B_z$ and $B_\phi \ll B_z$.  The idea is to then expand in the two-dimensional plane defined by a \SI{4.5}{\cm} circle in $r$ and $z$ at a single value of $\phi$.

\begin{align}
\label{eqn:field-expansion}
B_z & = \sum_{n=0}^{\infty} \rho^n[a_n \sin{\phi} + b_n \cos{\phi}] \\
B_r & = \sum_{n=1}^{\infty} \rho^n[c_n \sin{\phi} + d_n \cos{\phi}] \\
\end{align}

\noindent
A visualization of the terms in the multipole expansion is given in figure \ref{fig:field-example-multipoles}.  Each multipole highlights a possible symmetry of the field over the 2D azimuthal slice of the storage region. The magnetic field is characterized in terms of multipoles where the dipole should average to \bmagic and all other terms should be minimized.

\begin{figure}
\label{fig:field-example-multipoles}
\includegraphics[width=\linewidth]{fig/field-example-multipoles}
\caption{The first seven multipoles in the the field expansion, eqn. \ref{eqn:field-expansion}.  The first on the left is the dipole term which simply averages each point equally over the domain.  The next two are the normal and skew quadrupole which represent an inner to outer or top to bottom asymmetry in the field respectively.  The next four terms similarly represent symmetries of the field.  They are termed: normal sextupole, skew sextupole, normal octupole, and skew octupole.}
\end{figure}

\subsection{Determining $<\omega_p>$}

The magnetic field is measured and monitored with a suite of magnetometers, pNMR probes.  The probes are pulsed, and the signal is analyzed to produce a frequency value. The following chapter, \ref{ch:field-measurement} discusses the analysis in depth, but a cursory overview is still provided here for completeness.  

The field analysis combines pNMR measurements from three different major subsystems.  The trolley system carries a 2D array of probes to measure the field in the storage volume.  The trolley cannot run while the ring is accepting muons though.  The fixed probe system is a set of 378 stationary pNMR devices to measure field drift between trolley runs.  The final subsystem is the absolute calibration probes which are essentially independent devices from the rest of the pNMR probes.  The absolute calibration probes are used to correct for systematic effects that shift the proton precession frequency in the main probes.  

All the field measurements are combined to produce a storage field map.  The final deliverable for the field, $<\omega_p>$, must denote the average field experienced by muons though.  The muon distribution and the field values need to be integrating over the all the muons that produce usable events.  The final deliverable value from the field analysis in then given in equation \ref{eqn:field-omega-p-tilde}.

\begin{equation}
\label{eqn:field-omega-p-tilde}
<\omega_p> = \int M(\vec{r}, t) \cdot \omega_p(\vec{r}, t)\; dt\;dV
\end{equation}


\section{Spin Precession} \label{sec:spin-precession}

It is essential that the experiment store muons in stable orbits of the storage ring until they decay into electrons.  The muons come in polarized in the injection direction.  While propagating around the storage ring, the muons undergo standard Larmor precession and spin precession.  Eventually the muons decay into electrons and neutrinos as discussed in section \ref{sec:muon-attributes}. By good fortune of weak decays, the spin direction is correlated to the energy and direction of the decay electrons.

\subsection{Decay Characteristics}

The characteristics of the muon decay spectrum can be understood in the muon's rest frame, then boosted to the lab frame.  In the muon's rest frame, the muon decays with an energy ranging from 0 to half the rest mass of the muon depending on the orientation of the decay neutrinos.  The distribution of decays is described by the Michel spectrum depicted in figure \ref{fig:muon-decay-distributions}.  The figure also depicts the asymmetry of the decay electrons which is to say the fraction of electrons which decay with a momentum vector in the same direction as that of the muon spin vector.  In the rest frame it can be seen that accumulation of events with either less than half the total energy or energy greater than half the total possible energy would result in a signal correlated with the spin direction.\cite{e821-prd}

Applying similar logistics to boosted spectrum informs of the way the in which \gmtwo can work.  The key point is still to choose a domain which has an integral maximizing the spin-momentum correlation for the events.  The exact statistical figure of merit is 

\begin{equation}
\label{eqn:expt-figure-of-merit}
NA^2 = \int_{y_{thresh}}^{1} n(y) a^2(y) \;dy
\end{equation}

for an event counting analysis.  The ideal energy threshold in this case is $0.4\times$\pmagic$ = $ \SI{1.24}{\GeV}.  The connection of muon spin to the decay electron energy is now established, so the last step in measuring the spin precession requires a technique which measures the energy and the decay time of emitted electons.\cite{e821-prd}

\begin{figure}
\label{fig:muon-precession-fom}
\centering
\includegraphics[width=1.0\linewidth]{fig/muon-precession-fom}
\caption{The plot depicts an unnormalized probability distribution for the number of decay electrons in the boosted lab frame, and similarly the fractional asymmetry as a function of the energy where the energy is represented as a fraction of the magic muon momentum, \pmagic. Alongside the distributions, there is also a plot of $NA^2$ which is the statistical figure of merit for measuring decay events and maximizing the signal from the spin correlation.}
\end{figure}

\subsection{Electron Calorimetry}

The solution is a suite of calorimeters arranged around the inner radius of the storage ring.  When the muons decay, the electrons always have less energy than the magic momentum, and so must curl inward on a smaller orbit radius than the magic radius.  A smaller orbit that intersects with a calorimeter, at least the higher energy ones anyway. The trajectory of a typical decay electron is shown in figure \ref{fig:omega-a-decay-electron-trajectory}.

\begin{figure}
\label{fig:omega-a-decay-electron-trajectory}
\includegraphics[width=0.9\linewidth]{fig/omega-a-decay-electron-trajectory}
\caption{A typical decay electron trajectory.  The muon decays into an electron with lower momentum which by necessity takes a path with a smaller radius of curvature.  The electron curls inward into one of 24 calorimeter blocks around the inner radius of the storage ring.}
\end{figure}

The calorimeters establish a time and energy for each detected electron.  The device consists of an array of $PbF_2$ crystals each with a physically smaller array of geiger-like photon counting hardware called a silicon photomultiplier (SiPM).  The incoming electron produces copious \v{C}erenkov photons.  The photons collect at the opposite end of the crystal where they activate SiPM channels.  The subsequent pulse undergoes pulse fitting analysis routine to determine the energy and the time of the electron decay event.  The decay times for electrons with energy above the desired threshold are then histogrammed to produce the so-called ``wiggle'' plot.  The plot exhibits clear oscillations at \wa which through careful systematic analysis produces a number for the precession frequency.

\subsection{Determining $\omega_a$}

The signal for the muon precession frequency arises in histogramming electron detection events into time bins.  The muons undergo decay into electrons at an exponential rate, and the energy distribution of those electrons changes depending on the direction of the spin vector.  As the spin vector precesses, the number of decay electrons measured above threshold also fluctuates as shown in figure \ref{fig:omega-a-wiggle-plot}.  The basic model for the precession frequency signal is then given as 

\begin{equation}
\label{eqn:omega-a-signal}
N(t, E) = N_0(E) e^{-t/\tau_\mu} \left[ 1 + A(E) \cos(\omega_a t + \phi_0(E))\right]
\end{equation}

where $N_0$ is the total number of muons above threshold at time zero, $\tau_\mu$ is the muon lifetime, $A(E)$ is the total asymmetry of muons above threshold, and $\phi_0(E)$ is the phase of the spin vectors at the chosen start time for the fit.  A five parameter fit using equation \ref{omega-a-signal} yields the anomalous spin precession for muons.  The distribution and fit is depicted in figure \ref{fig:omega-a-wiggle-plot}.

\begin{figure}
\label{fig:omega-a-wiggle-plot}
\includegraphics[width=0.9\linewidth]{fig/omega-a-wiggle-plot}
\caption{The histogram is from the E989 TDR, reference\cite{e989-tdr}.  It shows the number of electron decay events above threshold as a function of time.  The oscillations seen on top of the exponential decay correlate to the spin precession of the muons as they propagate around the storage ring.  The number of high energy decays is enhanced as the spin aligns with the momentum and decreased when the spin anti-aligns with the momentum vector.}
\end{figure}


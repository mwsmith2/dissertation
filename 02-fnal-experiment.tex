% Chapter details the E989 g-2 Experiment
\chapter {E989: The Muon \gmtwo Experiment} \label{ch:expt}

\section{Overview} \label{sec:expt-overview}

\subsection{The Muon Anomaly}

The principal result from the E989 experiment can be represented by a single expression in which the two frequency parameters are quantities measured very precisely by the \mugmtwo experiment and all other parameters are fundamental quantities precisely measured in other experiments.

\begin{equation}
\label{eqn:g-2-result-1}
a_\mu = \frac{\omega_a}{\omega_p} \frac{\mu_p}{\mu_e} \\
\frac{m_\mu}{m_e} \frac{g_e}{2}
\end{equation}

\noindent
The parameter $\omega_a$ is the anomalous spin precession frequency for the muon, and $\omega_p$ is the Larmor frequency for a free proton.  Both frequencies are induced by the same magnetic field and thereby exactly correlated. The analysis technique for $\omega_a$ is described in more detail in section \ref{sec:spin-precession}.  The $\omega_p$ analysis methods are further discussed in section \ref{sec:magnetic-field} and further still in chapter \ref{ch:field}.  The CODATA values \cite{codata} for the magnetic moment and mass ratios and reference \cite{g-e-measurement} for the $g$-factor of the electron are given below.

\begin{align}
\mu_p/\mu_e & = -658.210\;684\;8(54) \\
m_\mu/m_e & = 206.768\;284\;3(52) \\
g_e/2 & = 1.001\;159\;652\;180\;73(28) \\
\end{align}

An alternative expression for $a_\mu$ can be derived by using the relation $m_\mu / m_e = (g_\mu \mu_\mu) / (g_e / \mu_e)$.  

\begin{equation}
\label{eqn:g-2-result-2}
a_\mu = \frac{\mathcal{R}}{\lambda^+ - \mathcal{R}}
\end{equation}

\noindent
where $\mathcal{R} = \omega_a / \omega_p$ and $\lambda^+ = \mu_{\mu^+} / \mu_p$  which is measured in a muonium hyperfine splitting experiment \cite{muonium-hyperfine}.  Using the CODATA values \cite{codata}, the ratio computes to

\begin{equation}
\label{eqn:muon-to-proton-mu-ratio}
\lambda^+ = 3.183\;345\;139(10).
\end{equation}

% In both expressions for $a_\mu$, the value for $\omega_p$ represents magnetic storage field.  It must be the average magnetic field sampled by the muons though, so the magnetic field storage data needs to be properly weighted by the muon particle trajectories.

% \noindent With all the measurement values included, the equations for the $a_\mu$ become

% \begin{align}
% \Rightarrow a_\mu  & = -136\;254.919\;130\;2(75) \cdot \mathcal{R}
% \Rightarrow a_\mu  & = \frac{\mathcal{R}
% \end{align}

\subsection{Precision Goals}

The overall precision goal on $a_\mu$ is \SI{140}{ppb}.  \SI{100}{ppb} of uncertainty is allotted for statistics.  The systematic uncertainty limit on $\omega_a$ is \SI{70}{ppb}, and the systematic uncertainty limit on $\omega_p$ is equally \SI{70}{ppb}.  The individual terms summed in quadrature make up the entire uncertainty budget for the FNAL \mugmtwo experiment, nearly a four-fold improvment upon the precision of the previous experiment at BNL.  The statistical uncertainty goal can be achieved through recording $1.8\times10^{11}$ physics events \cite{e989-tdr}.  The plan to reach systematic uncertainty limits includes broad experimental improvements over E821 in hardware systems, data quality, and analysis techniques.  A subset of the improvement plans for the $\omega_p$ uncertainties is discussed in chapter \ref{ch:field}.

\subsection{Stages of the Experiment}

The \mugmtwo experiment can be broken into several stages.  In the subsequent sections the following stages are discussed:

\begin{itemize}[noitemsep]
\item{Muon Production}
\item{The Magnetic Storage Ring}
\item{Muon Injection}
\item{The Magnetic Field}
\item{Spin Precession}
\end{itemize}

\noindent
In this document, a brief overview of information in the E989 Technical Design Report (TDR) is given. See the original document, reference \cite{e989-tdr}, for further detail. 

\section{Muon Production} \label{sec:muon-production}

The first stage in an experiment that measures a property of muons is of course producing muons. The full chain of muon production goes from protons to pions to muons (to electrons).  Expert personnel at Fermilab have developed a new beamline to deliver muons to the \gmtwo experiment using much of the existing particle production and accelerator infrastructure.  The structure of the beamline is illustrated in figure \ref{fig:fnal-beamline-diagram}.

% \begin{figure}
% \centering
% \includegraphics[width=0.9\linewidth]{fig/muon-production-beamline}
% \caption{The accelerator beamline that delivers muons for E989.  Protons are accelerated to \SI{8}{\GeV/c} and bunched in the Recycler Ring.  Then the protons hit the pion production target and lithium focusing lens.  The pions propagate down the M1-M2-M3 line to the Delivery Ring where muons are bunched and delivered to MC-1, the \mugmtwo experimental hall, at \SI{12}{\second^-1} \cite{e989-tdr}. \label{fig:muon-production-beamline}}
% \end{figure}

\begin{figure}
\includegraphics[width=0.9\linewidth]{fig/fnal-beamline-diagram}
\caption{
    The diagram depicts all relevant FNAL beamlines for \gmtwo.  Protons begin accelerating in the Linac, continue in the Booster, and enter the Recycler Ring. In the Recycler the protons are bunched into high intensity, small time windowed groups.  The protons exit the Recycler and propagate down the P1, P2, and M1 beamlines to the secondary production target.  Positive secondaries (including pions which later yield muons) at \SI{3.1}{\GeV/c} are focused and transported down the M2 and M3 beamlines to the Delivery Ring.  In the Delivery Ring, the bunch propagates long enough to develop a timing separation between protons and the muons now populating the beam.  With the timing separation, the protons can be dumped and the muons can be extracted to continue along M4 and M5 to the \gmtwo storage ring. 
    \label{fig:fnal-beamline-diagram}
}
\end{figure}

Much of the FNAL accelerator infrastructure is reused for protons in new beamlines for the Muon Campus.  The protons begin in the Linac and accelerate through Booster.  From there, the protons continue into the Recycler Ring where they are manipulated into the high intensity bunches with short timing structure for \gmtwo.  Each proton bunch contains $\mathcal{O}(10^{12})$ protons with \SI{8}{\GeV} kinetic energy in time windows less than \SI{90}{\nano \second}.

The protons then propagate from the Recycler to the AP0 target where they collide with the target.  The collision produces $\mathcal{O}(10^9)$ positive secondary particles of which many are pions.  The secondary particles are focused via an electrostatic lithium lens into a secondary beam which goes through a momentum filter shortly after focusing.  Momentum selection yields a beam of \SI{3.1}{\GeV/c} with a momentum spread of \SI{\pm 0.10}{\frac{\delta p}{p}}.  The secondary beam then proceeds through P1, P2, M1, M2, and M3 beamlines into the Delivery Ring.  

The goals in the Delivery Ring are twofold.  First, the beam cycles around the Delivery Ring to create a spatial separation between the pions/muons and the more massive protons (slightly lower velocity for the same momentum), so that the protons can be removed.  Secondly, essentially all pions undergo in-flight weak decay into muons, so the outgoing beam is a very pure muon beam.  Four orbits around the Delivery Ring are enough to achieve both goals.

After the Delivery Ring, the now muon beam is extracted onto the path toward the \mugmtwo storage ring. Through the pion decay process the high and low energy muons have a net spin polarization (as discussed in section \ref{sec:muon-attributes}), and the beamline design acceptance is narrow around the filtered secondary energy of \SI{3.1}{\GeV/c}.  The muons produced at \pmagic by the pion beam are forward decays and thereby achieve a net spin polarization of around \SI{95}{\percent}.  A bunch of muons produced in the beamline is referred to as a ``fill''. These fills deliver $\mathcal{O}(10^{4})$ muons to the storage ring at an average rate of \SI{12}{\second^{-1}} \cite{e989-tdr}.

\section{The Magnetic Storage Ring} \label{sec:storage-ring}
The magnetic storage ring is the hardware for creating the main magnetic dipole field for the experiment.  The major components include three superconducting NbTi/Cu coils, a dozen ``C''-shaped flux return yokes in the form of steel blocks and plates, 72 high-purity steel poles, and a built-in adjustment kit with more than 1,000 tunable knobs.  Figure \ref{fig:magnet-cross-section} depicts a cross-sectional view of the magnet.  The full assembly is around \SI{3}{\meter} tall, \SI{15}{\meter} wide, and weighs more than 700 tons \cite{e989-tdr}.  The storage ring was designed with the objective of creating a magnetic field of \bmagic that is as uniform as possible over a toroidal storage volume with a major radius of \rmagic and a minor radius of \SI{4.5}{\cm}.

\begin{figure}
\label{fig:magnet-cross-section}
\centering
\includegraphics[width=0.9\linewidth]{fig/magnet-cross-section}
\caption{The cross-sectional view of the storage ring.  The magnetic field is produced by currents in the three superconducting coils.  The field strength is primarily determined by the geometry for the flux capture in the ``C'' yoke and pole surfaces.  The field can be adjusted locally in azimuthal using a built-in shim kit that includes top hat shims, edge shims, and wedge shims.}
\end{figure}

The magnetic field itself is generated by driving a current of a current of $5176\;A$ through all three superconducting coils.  Two of the coils reside at $R_{inner} = \SI{6677}{\mm}$, inside the storage radius. And, one of the coils resides at a larger radius, $R_{outer} = \SI{7512}{\mm}$.  The outer coil contains twice as much superconducting material as the two inner coils to generate approximately equal magnetic fields from inner and outer coils.  The inner and outer coils run with opposing currents to create a nearly vertical B-field in the space between them.  The overall strength of the magnetic field depends on the amount of ferric material around the coils to contain magnetic flux and concentrate the magnetic field.  By design, the return yokes and pole pieces draw in magnetic flux at the operational current to produce a magnetic field close to \bmagic in the opening of the ``C''.  To first approximation the magnetic field uniformity depends on the geometric uniformity of the gap between the two pole surfaces.  The precision machined, high-purity pole pieces provide two extremely flat parallel surfaces for optimal field uniformity.  The final dipole magnetic field uniformity at BNL was around \SI{\pm50}{ppm} \cite{e821-prd}. An effort was made to recreate the final state of the magnet assembly after the storage ring was transported to FNAL and re-assembled.  When initially powered though, the magnetic field had variations of around \ppm{1400} as shown in figure \ref{fig:initial-field}.

\begin{figure}
\label{fig:initial-field}
\centering
\includegraphics[width=0.9\linewidth]{fig/initial-dipole-field.png}
\caption{The initial dipole (average at azimuthal point) magnetic field when the storage field was first fully measured at Fermilab.  The peak-to-peak variation was around \ppm{1400}.  The umbrella-like sub-structure occurs every 10 degrees and corresponds to the changing gap between the slightly curved pole pieces.  The horizontal bands represent the average field value and the uniformity goals set forth by the experiment.  The band formed by the two dotted lines around the average value indicates the experimental goal of \SI{\pm 25}{ppm}.}
\end{figure}

The last essential piece of the magnetic storage ring is the intrinsic shim kit.  Shims in this context are objects made of ferric material with full azimuthal coverage of the storage ring which can be adjusted to improve the magnetic field uniformity locally. A collection of edge shims, wedge shims, and top hats provides over 1,000 knobs to make azimuthally localized changes to the magnetic field.
Magnetic field shimming is discussed in further detail in chapter \ref{ch:shimming}.

\section{Muon Injection} \label{sec:muon-injection}
To store muons in the strong vertical magnetic field, they must be transported through the backleg, fringe field region of the magnet.  The non-linear natural path through this region is non-trivial to predict or manufacture in the storage ring.  The solution to simplify muon injection was to create a homogeneous magnetic field volume to provide a predictable path into the storage region. This field region is created with a cleverly designed magnet, called the inflector, which is able to achieve a strong, nearly uniform vertical field over a small volume.  It is designed to contain much of its flux from leaking and perturbing the nearby precision magnetic field region.  The flux capture in the initial design was insufficient and it was later improved by adding a superconducting shield to the outside.  The inflector was designed to be adjustable over small angles which allowed the apparatus to optimize the nearly straight injection of muons. See figure \ref{fig:expt-inflector} for an image of the inflector and the beam trajectory through the inflector. \cite{e989-tdr, e821-prd}

\begin{figure}
\label{fig:expt-inflector}
\includegraphics[width=0.9\linewidth]{fig/expt-inflector}
\caption{The inflector diagram on the left illustrates the relative position of the inflector as the bridge from essentially outside of the storage field through to within the storage volume.  The trajectory of the beamline through the inflector area is depicted in the right plot.}
\end{figure}

Another problem arises after the injected muons complete their first orbit around the storage ring.  Moving on circular orbits in a nearly uniform vertical magnetic field, the muons would hit the downstream end of the inflector as they return to the injection point.  In figure \ref{fig:expt-kicker-correction}, the red track illustrates the problem and the blue track illustrates a optimally shifted orbit achieved after a fast kicker magnet imparts the necessary angular shift onto the muon bunch.  The kicker would ideally produce a flat magnetic field around \SI{275}{\gauss} with very sharp rise and fall times, $\mathcal{O}(10\;ns)$.  The real pulse shape deviates from a perfect square pulse and re-kicks muons with the residual magnetic field, but a newly designed kicker will help to prevent double kicking muons as they return to the kicker region on their second orbit of the ring \cite{e989-tdr}.  

\begin{figure}
\centering
\includegraphics[width=0.6\linewidth]{fig/expt-kicker-correction}
\caption{The figure illustrates the orbital mismatch problem with muon injection.  The red track illustrates the problem where the muons collide with the injection point, and the blue track illustrates the good orbit achieved after a perfect kick. \label{fig:expt-kicker-correction}}
\end{figure}

\section{Magnetic Field} \label{sec:magnetic-field}

The magnetic field is of critical importance to the muon \gmtwo experiment.  A magnetic field of \bmagic puts muons with ``magic'' momentum, \pmagic, into uniform circular motion at the ``magic'' radius of \rmagic.  The prescribed parameters lock a fraction, $\mathcal{O}(0.03)$, of the injected muons in circular motion until they decay.  To first order, the value of the magnetic field directly affects the rate of muon spin precession and cyclotron frequency.  In this light, the average magnetic field must be very well measured, since it folds directly into the determination of \wa.  To second order the magnetic field couples to the symmetries in the muon beam influencing the beam dynamics of the stored muons.  These deviations in beam dynamics make the analysis which matches muon trajectories with the magnetic fields along them more difficult, and therefore add uncertainty to the determination of the expectation value for $\omega_p$.

\subsection{The Field Expansion}

The ideal magnetic field for the experiment is entirely vertical at \bmagic with no deviations.  However, Maxwell's equations do not permit such a perfect field within a finite amount of space.  Reality requires that the experiment consider the quality of the central field and the effects of perturbations to the field which is conveniently done using a decomposition.  The expansion in equation \ref{eqn:field-expansion} introduces a common set of functions used in discussing magnetic field perturbations.  The expansion assumes the domain where $B_r \ll B_z$ and $B_\phi \ll B_z$ effectively reducing the 3D problem to a 2D problem.  The idea is to then expand inside the two-dimensional plane defined by a \SI{4.5}{\cm} circle in $r$ and $z$ at a single value of $\phi$.

\begin{align}
\label{eqn:field-expansion}
\begin{split}
B_z & = \sum_{n=0}^{\infty} \rho^n[a_n \sin{n\phi} + b_n \cos{n\phi}] \\
B_r & = \sum_{n=1}^{\infty} \rho^n[c_n \sin{n\phi} + d_n \cos{n\phi}] \\
\end{split}
\end{align}

\noindent
A visualization of the terms in the multipole expansion is given in figure \ref{fig:field-example-multipoles}.  Each multipole highlights a possible symmetry of the field over the 2D azimuthal slice of the storage region. The magnetic field is characterized in terms of multipoles where the dipole should average to \bmagic and all other terms should be minimized.

\begin{figure}
\includegraphics[width=\linewidth]{fig/field-example-multipoles}
\caption{The first seven multipoles in the field expansion, equation \ref{eqn:field-expansion}.  The first on the left is the dipole term which simply averages each point equally over the domain.  The next two are the normal and skew quadrupole which represent an inner to outer or top to bottom asymmetry in the field respectively.  The next four terms similarly represent further symmetries of the field.  They are termed: normal sextupole, skew sextupole, normal octupole, and skew octupole. \label{fig:field-example-multipoles}}
\end{figure}

\subsection{Determining $\langle \omega_p \rangle$}

The magnetic field is measured and monitored with a suite of magnetometers, custom pNMR probes. Chapter \ref{ch:field} discusses the analysis in depth, but a cursory overview is still provided here for completeness.  The field analysis combines pNMR measurements from three different major subsystems.  The trolley system is a cylindrical aluminum shell equipped to travel through the muon storage volume and an array of pNMR probes to measure the magnetic field in 2D azimuthal slices.  A full set of measurements from the trolley system is combined to produce a magnetic field map as a function of position over the entire muon storage volume.  While the trolley resides in the storage volume, the ring cannot accept muons, so a set of 378 stationary pNMR devices outside the storage volume called the fixed probe system monitor the magnetic field drift between trolley runs.  The final subsystem is the absolute calibration probe system which includes several, essentially independent devices from the rest of the pNMR probes.  The absolute calibration probes are used to correct for systematic effects that shift the proton precession frequency in the trolley and fixed probes.

All the field measurements combine to produce the magnetic storage field as a function of time and position, the target quantity from the field measurements.  Expression \ref{eqn:g-2-result-1} and \ref{eqn:g-2-result-2} actually need $\langle \omega_p \rangle$ which represents the average field experienced by muons.  The muon distribution, $M(\vec{r}, t)$, and the field values need to be integrated over the all muons that contribute to determination of $\omega_a$.  An appropriate expression for $\langle \omega_p \rangle$ is then given in equation \ref{eqn:field-omega-p-tilde}.

\begin{equation}
\label{eqn:field-omega-p-tilde}
\langle \omega_p \rangle = \int M(\vec{r}, t) \, \omega_p(\vec{r}, t)\; dt\;dV
\end{equation}

\section{Spin Precession} \label{sec:spin-precession}

It is essential that the storage ring contain muons in stable orbits until they decay into electrons.  The muons come in polarized in the injection direction.  While propagating around the storage ring, the polarization vector of the muons undergoes spin precession.

\subsection{Decay Characteristics}

Eventually the muons decay into electrons and neutrinos as discussed in section \ref{sec:muon-attributes}. Due to maximal parity violation in weak decays, the spin direction is correlated to the energy and momentum direction of the decay electrons.  In the rest frame (see figure \ref{fig:muon-decay-rest-frame}) it can be seen that accumulation of events with energy either less than half or greater than half the total possible energy would result in a signal (anti-)correlated with the spin direction.  Applying similar logic to the boosted spectrum (see equation \ref{eqn:muon-decay-lab-frame} and figure \ref{fig:muon-precession-fom}) is a key insight into the way \gmtwo works.

\begin{align}
\label{eqn:muon-decay-lab-frame}
n_{lab}(y) & = \frac{-8 y^2 + y + 1}{4 y^2 - 5y - 5} & a_{lab}(y) & = \tfrac{1}{5}(y - 1) (4 y^2 - 5y - 5)
\end{align}

\noindent
The point is still to choose a domain which has an integral maximizing the spin-momentum correlation for the events.  The expression for the statistical figure of merit for an event counting analysis is

\begin{align}
\label{eqn:expt-figure-of-merit}
NA^2 = \int_{y_{thresh}}^{1} n(y) a^2(y) \;dy.
\end{align}

\noindent
The ideal energy threshold in this case is $0.4\times$\pmagic$ = $ \SI{1.24}{\GeV/c} as illustrated in figure \ref{fig:muon-precession-fom}.  The connection of muon spin to the decay electron energy is now established, so the last step in measuring the spin precession requires a technique to measure the energy and the birth time of emitted electons. \cite{e821-prd}

\begin{figure}
\centering
\includegraphics[width=1.0\linewidth]{fig/muon-precession-fom}
\caption{The plot depicts an unnormalized probability distribution for the number of decay electrons in the boosted lab frame, and similarly the fractional asymmetry as a function of the electron energy where the energy is represented as a fraction of the maximum possible electron energy from a muon momentum at \pmagic. Alongside the distributions, there is also a plot of $NA^2$ which is the statistical figure of merit for measuring decay events and maximizing the signal from the spin correlation. \label{fig:muon-precession-fom}}
\end{figure}

\subsection{Electron Calorimetry}

A suite of calorimeters arranged around the inner radius of the storage ring measure properties of the decay electrons.  The decay electrons always have less energy than the magic momentum, and so must curl inward on a smaller orbit radius than the storage orbit radius of the initial muon.  The trajectory of a typical decay electron is shown in figure \ref{fig:omega-a-decay-electron-trajectory}.  The trajectory of the electrons intersects with a calorimeter, at least the higher energy electrons anyway. The higher energy electrons have a wider trajectory similar to the parent muon and hit the calorimeters head on.  Lower energy electrons have a sharp trajectory which can curl inward between calorimeter stations.

\begin{figure}
\centering
\includegraphics[width=1.0\linewidth]{fig/omega-a-decay-electron-trajectory}
\caption{
    The figure shows a few typical decay electron trajectories.  The muon decays into an electron with lower momentum which by necessity takes a path with a smaller radius of curvature.  The electron curls inward into one of 24 calorimeter blocks around the inner radius of the storage ring. 
    \label{fig:omega-a-decay-electron-trajectory}
}
\end{figure}

The calorimeters establish a time and energy for each detected electron.  The device consists of a 6x9 array of $\mathrm{PbF_2}$ crystals each with a physically smaller array of geiger-like photon counting hardware called a silicon photomultiplier (SiPM).  Segmenting the detector decreases the likelihood of two decay events occurring simultaneously in the same measurement channel.  These double events are referred to as pileup and were a major source of uncertainty in E821.  The incoming electron produces thousands of \v{C}erenkov photons per \SI{}{\GeV}.  The photons collect at the opposite end of the crystal where they activate SiPM channels.  The subsequent pulse undergoes signal shaping, digitization, and a pulse fitting analysis routine to determine the energy and the time of the electron decay event.  

\subsection{Determining $\omega_a$}

The signal for the anomalous muon precession frequency manifests in histogramming electron detection events with energy above a cutoff threshold into time bins. The so-called ``wiggle'' plot exhibits clear oscillations which through careful systematic analysis result in a precise value for the anomalous precession frequency, $\omega_a$.  The muons undergo decay into electrons at an exponential rate, and the energy distribution of those electrons changes depending on the direction of the spin vector.  As the spin vector precesses, the number of decay electrons measured above threshold also fluctuates as shown in figure \ref{fig:omega-a-wiggle-plot}.  The basic model for the precession frequency signal is then given as 

\begin{equation}
\label{eqn:omega-a-signal}
N(t, E) = N_0(E) e^{-t/\tau_\mu} \left[ 1 + A(E) \cos(\omega_a t + \phi_0(E))\right]
\end{equation}

where $N_0$ is the total number of muons above threshold at time zero, $\tau_\mu$ is the effective muon lifetime, $A(E)$ is the total asymmetry of muons above threshold, and $\phi_0(E)$ is the phase of the spin vectors at $t_0$ for the fit.  A five parameter fit using equation \ref{eqn:omega-a-signal} yields the anomalous spin precession for muons.  The distribution and fit is depicted in figure \ref{fig:omega-a-wiggle-plot}.

\begin{figure}
\centering
\includegraphics[width=0.9\linewidth]{fig/omega-a-wiggle-plot}
\caption{
    The histogram is from the E989 TDR, reference\cite{e989-tdr}.  It shows the number of electron decay events above threshold as a function of time.  The oscillations seen on top of the exponential decay correlate to the spin precession of the muons as they propagate around the storage ring.  The number of high energy decays is enhanced as the spin aligns with the momentum and decreased when the spin anti-aligns with the momentum vector. 
    \label{fig:omega-a-wiggle-plot}
}
\end{figure}

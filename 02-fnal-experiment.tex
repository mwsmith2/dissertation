% Chapter details the methods and hardware used to create
% and measure the g-2 Magnetic Field.
\chapter {The Muon \gmtwo Experiment, E989}

The magnetic field is of critical importance to the muon \gmtwo experiment. To zero order, the experiment would not store muons with a magnetic field.  The magnetic field of \bmagic puts the muons of \rmagic into uniform circular motion at a radius of \pmagic.  The prescribed parameters lock a fraction of the injected muons in cyclotron motion until they decay.  To first order the value of the magnetic field directly affects the rate of spin precession, eqn. \ref{eqn:spin-precession}, and cyclotron frequency, eqn. \ref{eqn:cyclotron-freq}.  In this light, the average magnetic field must be well measured, since it folds into the final determination of \wa.  To second order the magnetic field couples to the symmetries in the muon beam influencing the beam dynamics of the stored muons.  These deviations in beam dynamics can cause the muons to experience a field that does not represent the average field and therefore alter the determination of the expectation value for \wa.

\begin{equation}
\vec{\omega}_s = -\frac{gq\vec{B}}{2m} - (1 - \gamma) \frac{q\vec{B}}{\gamma m}
\label{eqn:spin-precession}
\end{equation}

\begin{equation}
\vec{\omega}_c = -\frac{gq\vec{B}}{\gamma m}
\label{eqn:cyclotron-freq}
\end{equation}

\section{The Storage Ring}

The magnetic storage ring is the core hardware for creating the magnetic storage field.  The major components include three superconducting NbTi/Cu coils, a dozen "C"-shaped flux return yokes in the form of steel blocks and plates, 72 high purity steel poles, and nearly 1000 tunable shims.  The full assembly is quite impressive standing around three meter tall, fifteen meters wide, and weighing in at around 200 tons.  The storage ring was designed with the objective of creating a magnetic field of \bmagic that is uniform as possible over a toroidal muon storage volume with a major radius of \rmagic and a minor radius of $4.5\;mm$.

\begin{figure}
\includegraphics[width=0.9\linewidth]{fig/magnet-cross-section.png}
\caption{\todo{fill out}}
\label{fig:magnet-cross-section}
\end{figure}

The storage ring in its entirety laces all the components listed above together and more.  The largest division is the yoke section
\todo{finish ring assembly description + figures}

The field itself is produced by the superconducting coils.  A current of $5176\;A$ is driven through all three coils.  Two of the coils reside at a smaller radius than \rmagic and one of the coils resides at a larger radius (see figure \ref{fig:magnet-cross-section}).  The outer coil packs twice as much superconductor as the two inner coils to give balance.  The inner and outer coils run with opposing currents to create a mostly vertical B-field in the space between them.  By design, the return yoke pulls in magnetic flux to increase the strength of field.  The precision machined, high purity pole pieces provide an extremely flat surface to hone the field uniformity.  To first approximation the magnetic field uniformity depends on the uniformity of the gap between the two pole surfaces.  With some care to recreate the  conditions of the magnet assembly at the end of E821, the storage ring was assembled at Fermilab and powered to a field with variations of around $1400\; ppm$ as show in figure \ref{fig:initial-field}.

\begin{figure}
\includegraphics[width=0.9\linewidth]{fig/initial-dipole-field.png}
\caption{The initial dipole(mean) magnetic field when the storage field was first fully measured at Fermilab.  The peak-to-peak variation was around $1400\;ppm$.  The umbrella-like sub-structure occurs every 10 degrees and corresponds to the gaps between the pole pieces.}
\label{fig:initial-field}
\end{figure}

\section{The Field Expansion}

The ideal magnetic field for the experiment is perfectly vertical at \bmagic with no other components.  However, Maxwell and his renowned equations inform the researchers that this ideal field is not possible within a finite space.  Reality requires that the researchers consider the quality of the central field and the effects of perturbations to the field.  First, let's define the expansion to develop a common vocabulary for speaking about field perturbations, eqn. \ref{eqn:field-expansion}.

\begin{equation}
B_z = B_0 + \sum_{n=1}^{\infty} \rho^n[a_n \sin{\phi} + b_n \cos{\phi}]
B_r = \sum_{n=1}^{\infty} \rho^n[c_n \sin{\phi} + d_n \cos{\phi}]
\label{eqn:field-expansion}
\end{equation}

The main field






